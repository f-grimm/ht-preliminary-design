%% Generated by Sphinx.
\def\sphinxdocclass{report}
\documentclass[letterpaper,10pt,english]{sphinxmanual}
\ifdefined\pdfpxdimen
   \let\sphinxpxdimen\pdfpxdimen\else\newdimen\sphinxpxdimen
\fi \sphinxpxdimen=.75bp\relax
\ifdefined\pdfimageresolution
    \pdfimageresolution= \numexpr \dimexpr1in\relax/\sphinxpxdimen\relax
\fi
%% let collapsible pdf bookmarks panel have high depth per default
\PassOptionsToPackage{bookmarksdepth=5}{hyperref}

\PassOptionsToPackage{warn}{textcomp}
\usepackage[utf8]{inputenc}
\ifdefined\DeclareUnicodeCharacter
% support both utf8 and utf8x syntaxes
  \ifdefined\DeclareUnicodeCharacterAsOptional
    \def\sphinxDUC#1{\DeclareUnicodeCharacter{"#1}}
  \else
    \let\sphinxDUC\DeclareUnicodeCharacter
  \fi
  \sphinxDUC{00A0}{\nobreakspace}
  \sphinxDUC{2500}{\sphinxunichar{2500}}
  \sphinxDUC{2502}{\sphinxunichar{2502}}
  \sphinxDUC{2514}{\sphinxunichar{2514}}
  \sphinxDUC{251C}{\sphinxunichar{251C}}
  \sphinxDUC{2572}{\textbackslash}
\fi
\usepackage{cmap}
\usepackage[T1]{fontenc}
\usepackage{amsmath,amssymb,amstext}
\usepackage{babel}



\usepackage{tgtermes}
\usepackage{tgheros}
\renewcommand{\ttdefault}{txtt}



\usepackage[Bjarne]{fncychap}
\usepackage{sphinx}

\fvset{fontsize=auto}
\usepackage{geometry}


% Include hyperref last.
\usepackage{hyperref}
% Fix anchor placement for figures with captions.
\usepackage{hypcap}% it must be loaded after hyperref.
% Set up styles of URL: it should be placed after hyperref.
\urlstyle{same}

\addto\captionsenglish{\renewcommand{\contentsname}{Contents:}}

\usepackage{sphinxmessages}
\setcounter{tocdepth}{1}



\title{Preliminary Design}
\date{Apr 27, 2022}
\release{}
\author{Fabian Grimm, Sumeet Kumar}
\newcommand{\sphinxlogo}{\vbox{}}
\renewcommand{\releasename}{}
\makeindex
\begin{document}

\pagestyle{empty}
\sphinxmaketitle
\pagestyle{plain}
\sphinxtableofcontents
\pagestyle{normal}
\phantomsection\label{\detokenize{index::doc}}



\chapter{Input}
\label{\detokenize{input:input}}\label{\detokenize{input::doc}}
\sphinxAtStartPar
Missions and helicopter configurations are provided as \sphinxcode{\sphinxupquote{.yaml}}\sphinxhyphen{}files in ‘\sphinxstyleemphasis{python/data}’. Examples are given below.


\section{Missions}
\label{\detokenize{input:missions}}
\begin{sphinxVerbatim}[commandchars=\\\{\}]
\PYG{n+nt}{Name}\PYG{p}{:} \PYG{l+lScalar+lScalarPlain}{name}

\PYG{n+nt}{Mission}\PYG{p}{:}

    \PYG{c+c1}{\PYGZsh{} Constant within segment}
    \PYG{n+nt}{Duration}\PYG{p}{:} \PYG{p+pIndicator}{[}\PYG{n+nv}{0.1}\PYG{p+pIndicator}{,} \PYG{n+nv}{0.1}\PYG{p+pIndicator}{]} \PYG{c+c1}{\PYGZsh{} h}
    \PYG{n+nt}{Payload}\PYG{p}{:} \PYG{p+pIndicator}{[}\PYG{n+nv}{0}\PYG{p+pIndicator}{,} \PYG{n+nv}{100}\PYG{p+pIndicator}{]} \PYG{c+c1}{\PYGZsh{} kg}
    \PYG{n+nt}{Crew mass}\PYG{p}{:} \PYG{p+pIndicator}{[}\PYG{n+nv}{0}\PYG{p+pIndicator}{,} \PYG{n+nv}{0}\PYG{p+pIndicator}{]} \PYG{c+c1}{\PYGZsh{} kg}
    \PYG{n+nt}{Flight speed}\PYG{p}{:} \PYG{p+pIndicator}{[}\PYG{n+nv}{10}\PYG{p+pIndicator}{,} \PYG{n+nv}{10}\PYG{p+pIndicator}{]} \PYG{c+c1}{\PYGZsh{} m/s}
    \PYG{n+nt}{Temperature offset}\PYG{p}{:} \PYG{p+pIndicator}{[}\PYG{n+nv}{0}\PYG{p+pIndicator}{,} \PYG{n+nv}{0}\PYG{p+pIndicator}{]} \PYG{c+c1}{\PYGZsh{} °C (deviation from ISA)}
    \PYG{n+nt}{Gravity}\PYG{p}{:} \PYG{p+pIndicator}{[}\PYG{n+nv}{9.76}\PYG{p+pIndicator}{,} \PYG{n+nv}{9.80}\PYG{p+pIndicator}{]} \PYG{c+c1}{\PYGZsh{} m/s\(\sp{\text{2}}\) (optional, default 9.81)}

    \PYG{c+c1}{\PYGZsh{} Linear with defined points between segments}
    \PYG{n+nt}{Height}\PYG{p}{:} \PYG{p+pIndicator}{[}\PYG{n+nv}{0}\PYG{p+pIndicator}{,} \PYG{n+nv}{0}\PYG{p+pIndicator}{,} \PYG{n+nv}{0}\PYG{p+pIndicator}{]} \PYG{c+c1}{\PYGZsh{} m}
\end{sphinxVerbatim}


\section{Configurations}
\label{\detokenize{input:configurations}}
\begin{sphinxVerbatim}[commandchars=\\\{\}]
\PYG{n+nt}{Name}\PYG{p}{:} \PYG{l+lScalar+lScalarPlain}{name}

\PYG{n+nt}{Main rotor}\PYG{p}{:}

    \PYG{n+nt}{Number of blades}\PYG{p}{:} \PYG{l+lScalar+lScalarPlain}{4}
    \PYG{n+nt}{Kappa}\PYG{p}{:} \PYG{l+lScalar+lScalarPlain}{1.15} \PYG{c+c1}{\PYGZsh{} (real/ideal induced power)}
    \PYG{n+nt}{Zero\PYGZhy{}lift drag coeff.}\PYG{p}{:} \PYG{l+lScalar+lScalarPlain}{0.011} \PYG{c+c1}{\PYGZsh{} (avg., for profile power)}
    \PYG{n+nt}{Tip velocity}\PYG{p}{:} \PYG{l+lScalar+lScalarPlain}{213} \PYG{c+c1}{\PYGZsh{} m/s}
    \PYG{n+nt}{Installation height}\PYG{p}{:} \PYG{l+lScalar+lScalarPlain}{2.5} \PYG{c+c1}{\PYGZsh{} m (for in\PYGZhy{}ground effect)}

\PYG{n+nt}{Tail rotor}\PYG{p}{:}

    \PYG{c+c1}{\PYGZsh{} Other rotor attributes are possible, but not used.}
    \PYG{n+nt}{Power fraction}\PYG{p}{:} \PYG{l+lScalar+lScalarPlain}{0.05}

\PYG{n+nt}{Engines}\PYG{p}{:}

    \PYG{n+nt}{Number of engines}\PYG{p}{:} \PYG{l+lScalar+lScalarPlain}{1}
    \PYG{n+nt}{Power available}\PYG{p}{:} \PYG{l+lScalar+lScalarPlain}{500\PYGZus{}000} \PYG{c+c1}{\PYGZsh{} W (per engine)}
    \PYG{n+nt}{SFC}\PYG{p}{:} \PYG{l+lScalar+lScalarPlain}{0.00035} \PYG{c+c1}{\PYGZsh{} kg/Wh}
    \PYG{n+nt}{A}\PYG{p}{:} \PYG{l+lScalar+lScalarPlain}{40} \PYG{c+c1}{\PYGZsh{} kg/h (A, B are optional, for power\PYGZhy{}dependent SFC)}
    \PYG{n+nt}{B}\PYG{p}{:} \PYG{l+lScalar+lScalarPlain}{0.00025} \PYG{c+c1}{\PYGZsh{} kg/Wh}

\PYG{n+nt}{Fuselage}\PYG{p}{:}

    \PYG{n+nt}{Download factor}\PYG{p}{:} \PYG{l+lScalar+lScalarPlain}{0.05} \PYG{c+c1}{\PYGZsh{} (thrust increase due to fuselage download)}
    \PYG{n+nt}{Drag area}\PYG{p}{:} \PYG{l+lScalar+lScalarPlain}{1.5} \PYG{c+c1}{\PYGZsh{} m\(\sp{\text{2}}\)}
    \PYG{n+nt}{Number of seats}\PYG{p}{:} \PYG{l+lScalar+lScalarPlain}{3} \PYG{c+c1}{\PYGZsh{} (used in empty weight estimation)}

\PYG{n+nt}{Landing gear}\PYG{p}{:}

    \PYG{n+nt}{Type}\PYG{p}{:} \PYG{l+lScalar+lScalarPlain}{Rigid}\PYG{l+lScalar+lScalarPlain}{ }\PYG{l+lScalar+lScalarPlain}{wheels} \PYG{c+c1}{\PYGZsh{} \PYGZob{}Skids, Retractable wheels, Rigid wheels\PYGZcb{}}
    \PYG{n+nt}{Number of legs}\PYG{p}{:} \PYG{l+lScalar+lScalarPlain}{2} \PYG{c+c1}{\PYGZsh{} (used in empty weight estimation)}

\PYG{n+nt}{Misc}\PYG{p}{:}

    \PYG{n+nt}{Empty weight ratio}\PYG{p}{:} \PYG{l+lScalar+lScalarPlain}{0.5} \PYG{c+c1}{\PYGZsh{} (EW / MTOW)}
    \PYG{n+nt}{Accessory power}\PYG{p}{:} \PYG{l+lScalar+lScalarPlain}{48\PYGZus{}000} \PYG{c+c1}{\PYGZsh{} W [p. 271]}
    \PYG{n+nt}{Transmission efficiency}\PYG{p}{:} \PYG{l+lScalar+lScalarPlain}{0.98} \PYG{c+c1}{\PYGZsh{} (default 1.0)}
    \PYG{n+nt}{MTOW}\PYG{p}{:} \PYG{l+lScalar+lScalarPlain}{1500} \PYG{c+c1}{\PYGZsh{} kg (not needed for sizing)}
\end{sphinxVerbatim}


\chapter{Structure}
\label{\detokenize{structure:structure}}\label{\detokenize{structure::doc}}

\section{UML}
\label{\detokenize{structure:uml}}
\begin{figure}[htbp]
\centering

\noindent\sphinxincludegraphics[width=800\sphinxpxdimen]{{UML_diagram}.png}
\end{figure}


\chapter{Modules}
\label{\detokenize{modules/modules:modules}}\label{\detokenize{modules/modules::doc}}

\section{Aircraft}
\label{\detokenize{modules/aircraft:aircraft}}\label{\detokenize{modules/aircraft::doc}}

\subsection{Thrust and angle of attack}
\label{\detokenize{modules/aircraft:thrust-and-angle-of-attack}}
\begin{figure}[htbp]
\centering
\capstart

\noindent\sphinxincludegraphics[width=400\sphinxpxdimen]{{aircraft}.png}
\caption{Definition of the climb angle and angle of attack.}\label{\detokenize{modules/aircraft:id1}}\end{figure}

\sphinxAtStartPar
Thrust vector:
\begin{equation*}
\begin{split}\vec{T} = -\vec{W} - \vec{D} = - \begin{bmatrix}0 \\ - mg\end{bmatrix} - \begin{bmatrix}D \cos \gamma \\ - D \sin \gamma \end{bmatrix}\end{split}
\end{equation*}
\sphinxAtStartPar
Angle of attack:
\begin{equation*}
\begin{split}\alpha = \frac{\pi}{2} - \theta\end{split}
\end{equation*}
\sphinxAtStartPar
with
\begin{equation*}
\begin{split}\theta = \arccos(\bar{\vec{D}} \cdot \bar{\vec{T}}), \quad \bar{\vec{D}} = \frac{\vec{D}}{ ||\vec{D}|| } , \quad \bar{\vec{T}} = \frac{\vec{T}}{ ||\vec{T}|| }\end{split}
\end{equation*}
\begin{figure}[htbp]
\centering
\capstart

\noindent\sphinxincludegraphics[width=270\sphinxpxdimen]{{aoa}.png}
\caption{Force balance between thrust, drag, and aircraft weight.}\label{\detokenize{modules/aircraft:id2}}\end{figure}
\phantomsection\label{\detokenize{modules/aircraft:module-aircraft}}\index{module@\spxentry{module}!aircraft@\spxentry{aircraft}}\index{aircraft@\spxentry{aircraft}!module@\spxentry{module}}\index{Aircraft (class in aircraft)@\spxentry{Aircraft}\spxextra{class in aircraft}}

\begin{fulllineitems}
\phantomsection\label{\detokenize{modules/aircraft:aircraft.Aircraft}}\pysiglinewithargsret{\sphinxbfcode{\sphinxupquote{class\DUrole{w}{  }}}\sphinxcode{\sphinxupquote{aircraft.}}\sphinxbfcode{\sphinxupquote{Aircraft}}}{\emph{\DUrole{n}{filename}\DUrole{p}{:}\DUrole{w}{  }\DUrole{n}{str}}}{}
\sphinxAtStartPar
Bases: \sphinxcode{\sphinxupquote{object}}

\sphinxAtStartPar
Vertical flight aircraft.
\index{name (aircraft.Aircraft attribute)@\spxentry{name}\spxextra{aircraft.Aircraft attribute}}

\begin{fulllineitems}
\phantomsection\label{\detokenize{modules/aircraft:aircraft.Aircraft.name}}\pysigline{\sphinxbfcode{\sphinxupquote{name}}}
\sphinxAtStartPar
Name of the aircraft.
\begin{quote}\begin{description}
\item[{Type}] \leavevmode
\sphinxAtStartPar
str

\end{description}\end{quote}

\end{fulllineitems}

\index{mtow (aircraft.Aircraft attribute)@\spxentry{mtow}\spxextra{aircraft.Aircraft attribute}}

\begin{fulllineitems}
\phantomsection\label{\detokenize{modules/aircraft:aircraft.Aircraft.mtow}}\pysigline{\sphinxbfcode{\sphinxupquote{mtow}}}
\sphinxAtStartPar
Maximum take\sphinxhyphen{}off weight; kg
\begin{quote}\begin{description}
\item[{Type}] \leavevmode
\sphinxAtStartPar
float

\end{description}\end{quote}

\end{fulllineitems}

\index{empty\_weight\_ratio (aircraft.Aircraft attribute)@\spxentry{empty\_weight\_ratio}\spxextra{aircraft.Aircraft attribute}}

\begin{fulllineitems}
\phantomsection\label{\detokenize{modules/aircraft:aircraft.Aircraft.empty_weight_ratio}}\pysigline{\sphinxbfcode{\sphinxupquote{empty\_weight\_ratio}}}
\sphinxAtStartPar
Ratio of empty weight to maximum take\sphinxhyphen{}off weight.
\begin{quote}\begin{description}
\item[{Type}] \leavevmode
\sphinxAtStartPar
float

\end{description}\end{quote}

\end{fulllineitems}

\index{special\_equipment (aircraft.Aircraft attribute)@\spxentry{special\_equipment}\spxextra{aircraft.Aircraft attribute}}

\begin{fulllineitems}
\phantomsection\label{\detokenize{modules/aircraft:aircraft.Aircraft.special_equipment}}\pysigline{\sphinxbfcode{\sphinxupquote{special\_equipment}}}
\sphinxAtStartPar
Special equipment mass; kg
\begin{quote}\begin{description}
\item[{Type}] \leavevmode
\sphinxAtStartPar
float

\end{description}\end{quote}

\end{fulllineitems}

\index{accessory\_power (aircraft.Aircraft attribute)@\spxentry{accessory\_power}\spxextra{aircraft.Aircraft attribute}}

\begin{fulllineitems}
\phantomsection\label{\detokenize{modules/aircraft:aircraft.Aircraft.accessory_power}}\pysigline{\sphinxbfcode{\sphinxupquote{accessory\_power}}}
\sphinxAtStartPar
Accessory power; W
\begin{quote}\begin{description}
\item[{Type}] \leavevmode
\sphinxAtStartPar
float

\end{description}\end{quote}

\end{fulllineitems}

\index{eta\_transmission (aircraft.Aircraft attribute)@\spxentry{eta\_transmission}\spxextra{aircraft.Aircraft attribute}}

\begin{fulllineitems}
\phantomsection\label{\detokenize{modules/aircraft:aircraft.Aircraft.eta_transmission}}\pysigline{\sphinxbfcode{\sphinxupquote{eta\_transmission}}}
\sphinxAtStartPar
Efficiency of the transmission.
\begin{quote}\begin{description}
\item[{Type}] \leavevmode
\sphinxAtStartPar
float

\end{description}\end{quote}

\end{fulllineitems}

\index{get\_thrust\_and\_alpha() (aircraft.Aircraft method)@\spxentry{get\_thrust\_and\_alpha()}\spxextra{aircraft.Aircraft method}}

\begin{fulllineitems}
\phantomsection\label{\detokenize{modules/aircraft:aircraft.Aircraft.get_thrust_and_alpha}}\pysiglinewithargsret{\sphinxbfcode{\sphinxupquote{get\_thrust\_and\_alpha}}}{\emph{\DUrole{n}{gross\_weight}}, \emph{\DUrole{n}{drag}}, \emph{\DUrole{n}{gamma}}, \emph{\DUrole{n}{gravity}}}{}
\sphinxAtStartPar
Determine the thrust and angle of attack by means of a force balance
with drag and weight vectors (isolated rotor).
\begin{quote}\begin{description}
\item[{Parameters}] \leavevmode\begin{itemize}
\item {} 
\sphinxAtStartPar
\sphinxstyleliteralstrong{\sphinxupquote{gross\_weight}} (\sphinxstyleliteralemphasis{\sphinxupquote{float}}) \textendash{} Aircraft mass; kg

\item {} 
\sphinxAtStartPar
\sphinxstyleliteralstrong{\sphinxupquote{drag}} (\sphinxstyleliteralemphasis{\sphinxupquote{float}}) \textendash{} Drag force acting on the aircraft; N

\item {} 
\sphinxAtStartPar
\sphinxstyleliteralstrong{\sphinxupquote{gamma}} (\sphinxstyleliteralemphasis{\sphinxupquote{float}}) \textendash{} Climb angle; rad

\item {} 
\sphinxAtStartPar
\sphinxstyleliteralstrong{\sphinxupquote{gravity}} (\sphinxstyleliteralemphasis{\sphinxupquote{float}}) \textendash{} Gravitational acceleration; m/s\(\sp{\text{2}}\)

\end{itemize}

\item[{Returns}] \leavevmode
\sphinxAtStartPar
\begin{itemize}
\item {} 
\sphinxAtStartPar
\sphinxstyleemphasis{float} \textendash{} Thrust; N

\item {} 
\sphinxAtStartPar
\sphinxstyleemphasis{float} \textendash{} Angle of attack; rad

\end{itemize}


\end{description}\end{quote}

\end{fulllineitems}


\end{fulllineitems}



\section{Engines}
\label{\detokenize{modules/engines:module-engines}}\label{\detokenize{modules/engines:engines}}\label{\detokenize{modules/engines::doc}}\index{module@\spxentry{module}!engines@\spxentry{engines}}\index{engines@\spxentry{engines}!module@\spxentry{module}}\index{Engines (class in engines)@\spxentry{Engines}\spxextra{class in engines}}

\begin{fulllineitems}
\phantomsection\label{\detokenize{modules/engines:engines.Engines}}\pysiglinewithargsret{\sphinxbfcode{\sphinxupquote{class\DUrole{w}{  }}}\sphinxcode{\sphinxupquote{engines.}}\sphinxbfcode{\sphinxupquote{Engines}}}{\emph{\DUrole{n}{engine\_data}\DUrole{p}{:}\DUrole{w}{  }\DUrole{n}{dict}}}{}
\sphinxAtStartPar
Bases: \sphinxcode{\sphinxupquote{object}}

\sphinxAtStartPar
Turboshaft engines as aircraft components.
\index{number\_of\_engines (engines.Engines attribute)@\spxentry{number\_of\_engines}\spxextra{engines.Engines attribute}}

\begin{fulllineitems}
\phantomsection\label{\detokenize{modules/engines:engines.Engines.number_of_engines}}\pysigline{\sphinxbfcode{\sphinxupquote{number\_of\_engines}}}
\sphinxAtStartPar
Number of engines.
\begin{quote}\begin{description}
\item[{Type}] \leavevmode
\sphinxAtStartPar
int

\end{description}\end{quote}

\end{fulllineitems}

\index{power\_available (engines.Engines attribute)@\spxentry{power\_available}\spxextra{engines.Engines attribute}}

\begin{fulllineitems}
\phantomsection\label{\detokenize{modules/engines:engines.Engines.power_available}}\pysigline{\sphinxbfcode{\sphinxupquote{power\_available}}}
\sphinxAtStartPar
Power available per engine; W
\begin{quote}\begin{description}
\item[{Type}] \leavevmode
\sphinxAtStartPar
float

\end{description}\end{quote}

\end{fulllineitems}

\index{sfc (engines.Engines attribute)@\spxentry{sfc}\spxextra{engines.Engines attribute}}

\begin{fulllineitems}
\phantomsection\label{\detokenize{modules/engines:engines.Engines.sfc}}\pysigline{\sphinxbfcode{\sphinxupquote{sfc}}}
\sphinxAtStartPar
Specific fuel consumption; kg/Wh
\begin{quote}\begin{description}
\item[{Type}] \leavevmode
\sphinxAtStartPar
float

\end{description}\end{quote}

\end{fulllineitems}



\begin{fulllineitems}
\pysigline{\sphinxbfcode{\sphinxupquote{a,~b}}}
\sphinxAtStartPar
Engine parameters A and B, determining the fuel consumption.
\begin{quote}\begin{description}
\item[{Type}] \leavevmode
\sphinxAtStartPar
float

\end{description}\end{quote}

\end{fulllineitems}

\index{get\_sfc() (engines.Engines method)@\spxentry{get\_sfc()}\spxextra{engines.Engines method}}

\begin{fulllineitems}
\phantomsection\label{\detokenize{modules/engines:engines.Engines.get_sfc}}\pysiglinewithargsret{\sphinxbfcode{\sphinxupquote{get\_sfc}}}{\emph{\DUrole{n}{temperature\_ratio}}, \emph{\DUrole{n}{pressure\_ratio}}, \emph{\DUrole{n}{power}}}{}
\sphinxAtStartPar
Calculate the power\sphinxhyphen{}dependent specific fuel consumption based on the
engine parameters A and B. {[}p.310{]}
\begin{quote}\begin{description}
\item[{Parameters}] \leavevmode\begin{itemize}
\item {} 
\sphinxAtStartPar
\sphinxstyleliteralstrong{\sphinxupquote{temperature\_ratio}} (\sphinxstyleliteralemphasis{\sphinxupquote{float}}) \textendash{} Temperature ratio relative to mean sea level.

\item {} 
\sphinxAtStartPar
\sphinxstyleliteralstrong{\sphinxupquote{pressure\_ratio}} (\sphinxstyleliteralemphasis{\sphinxupquote{float}}) \textendash{} Pressure ratio relative to mean sea level.

\item {} 
\sphinxAtStartPar
\sphinxstyleliteralstrong{\sphinxupquote{power}} (\sphinxstyleliteralemphasis{\sphinxupquote{float}}) \textendash{} Total power; W

\end{itemize}

\item[{Returns}] \leavevmode
\sphinxAtStartPar
Specific fuel consumption; kg/Wh

\item[{Return type}] \leavevmode
\sphinxAtStartPar
float

\end{description}\end{quote}

\end{fulllineitems}


\end{fulllineitems}



\section{Fuselage}
\label{\detokenize{modules/fuselage:module-fuselage}}\label{\detokenize{modules/fuselage:fuselage}}\label{\detokenize{modules/fuselage::doc}}\index{module@\spxentry{module}!fuselage@\spxentry{fuselage}}\index{fuselage@\spxentry{fuselage}!module@\spxentry{module}}\index{Fuselage (class in fuselage)@\spxentry{Fuselage}\spxextra{class in fuselage}}

\begin{fulllineitems}
\phantomsection\label{\detokenize{modules/fuselage:fuselage.Fuselage}}\pysiglinewithargsret{\sphinxbfcode{\sphinxupquote{class\DUrole{w}{  }}}\sphinxcode{\sphinxupquote{fuselage.}}\sphinxbfcode{\sphinxupquote{Fuselage}}}{\emph{\DUrole{n}{fuselage\_data}\DUrole{p}{:}\DUrole{w}{  }\DUrole{n}{dict}}}{}
\sphinxAtStartPar
Bases: \sphinxcode{\sphinxupquote{object}}

\sphinxAtStartPar
Fuselage as an aircraft component.
\index{drag\_area (fuselage.Fuselage attribute)@\spxentry{drag\_area}\spxextra{fuselage.Fuselage attribute}}

\begin{fulllineitems}
\phantomsection\label{\detokenize{modules/fuselage:fuselage.Fuselage.drag_area}}\pysigline{\sphinxbfcode{\sphinxupquote{drag\_area}}}
\sphinxAtStartPar
Drag area of the fuselage.
\begin{quote}\begin{description}
\item[{Type}] \leavevmode
\sphinxAtStartPar
float

\end{description}\end{quote}

\end{fulllineitems}

\index{download\_stationary (fuselage.Fuselage attribute)@\spxentry{download\_stationary}\spxextra{fuselage.Fuselage attribute}}

\begin{fulllineitems}
\phantomsection\label{\detokenize{modules/fuselage:fuselage.Fuselage.download_stationary}}\pysigline{\sphinxbfcode{\sphinxupquote{download\_stationary}}}
\sphinxAtStartPar
Relative increase in thrust due to obstructed downwash (stationary).
\begin{quote}\begin{description}
\item[{Type}] \leavevmode
\sphinxAtStartPar
float

\end{description}\end{quote}

\end{fulllineitems}

\index{number\_of\_seats (fuselage.Fuselage attribute)@\spxentry{number\_of\_seats}\spxextra{fuselage.Fuselage attribute}}

\begin{fulllineitems}
\phantomsection\label{\detokenize{modules/fuselage:fuselage.Fuselage.number_of_seats}}\pysigline{\sphinxbfcode{\sphinxupquote{number\_of\_seats}}}
\sphinxAtStartPar
Number of seats.
\begin{quote}\begin{description}
\item[{Type}] \leavevmode
\sphinxAtStartPar
int

\end{description}\end{quote}

\end{fulllineitems}

\index{get\_download\_factor\_in\_flight() (fuselage.Fuselage method)@\spxentry{get\_download\_factor\_in\_flight()}\spxextra{fuselage.Fuselage method}}

\begin{fulllineitems}
\phantomsection\label{\detokenize{modules/fuselage:fuselage.Fuselage.get_download_factor_in_flight}}\pysiglinewithargsret{\sphinxbfcode{\sphinxupquote{get\_download\_factor\_in\_flight}}}{\emph{\DUrole{n}{advance\_ratio}}}{}
\sphinxAtStartPar
Calculate the download factor in forward flight, assuming a linear
decline until advance ratio 0.5.
\begin{quote}\begin{description}
\item[{Parameters}] \leavevmode
\sphinxAtStartPar
\sphinxstyleliteralstrong{\sphinxupquote{advance\_ratio}} (\sphinxstyleliteralemphasis{\sphinxupquote{float}}) \textendash{} Advance ratio.

\item[{Returns}] \leavevmode
\sphinxAtStartPar
Download factor in forward flight.

\item[{Return type}] \leavevmode
\sphinxAtStartPar
float

\end{description}\end{quote}

\end{fulllineitems}

\index{get\_fuselage\_drag() (fuselage.Fuselage method)@\spxentry{get\_fuselage\_drag()}\spxextra{fuselage.Fuselage method}}

\begin{fulllineitems}
\phantomsection\label{\detokenize{modules/fuselage:fuselage.Fuselage.get_fuselage_drag}}\pysiglinewithargsret{\sphinxbfcode{\sphinxupquote{get\_fuselage\_drag}}}{\emph{\DUrole{n}{density}}, \emph{\DUrole{n}{flight\_speed}}}{}
\sphinxAtStartPar
Calculate the fuselage drag based on a constant drag area.
\begin{quote}\begin{description}
\item[{Parameters}] \leavevmode\begin{itemize}
\item {} 
\sphinxAtStartPar
\sphinxstyleliteralstrong{\sphinxupquote{density}} (\sphinxstyleliteralemphasis{\sphinxupquote{float}}) \textendash{} Density of the surrounding air; kg/m\(\sp{\text{3}}\)

\item {} 
\sphinxAtStartPar
\sphinxstyleliteralstrong{\sphinxupquote{flight\_speed}} (\sphinxstyleliteralemphasis{\sphinxupquote{float}}) \textendash{} Flight speed; m/s

\end{itemize}

\item[{Returns}] \leavevmode
\sphinxAtStartPar
Fuselage drag; N

\item[{Return type}] \leavevmode
\sphinxAtStartPar
float

\end{description}\end{quote}

\end{fulllineitems}

\index{get\_parasite\_power() (fuselage.Fuselage method)@\spxentry{get\_parasite\_power()}\spxextra{fuselage.Fuselage method}}

\begin{fulllineitems}
\phantomsection\label{\detokenize{modules/fuselage:fuselage.Fuselage.get_parasite_power}}\pysiglinewithargsret{\sphinxbfcode{\sphinxupquote{get\_parasite\_power}}}{\emph{\DUrole{n}{density}}, \emph{\DUrole{n}{flight\_speed}}}{}
\sphinxAtStartPar
Calculate the parasite power due to fuselage drag in forward flight.
\begin{quote}\begin{description}
\item[{Parameters}] \leavevmode\begin{itemize}
\item {} 
\sphinxAtStartPar
\sphinxstyleliteralstrong{\sphinxupquote{density}} (\sphinxstyleliteralemphasis{\sphinxupquote{float}}) \textendash{} Density of the surrounding air; kg/m\(\sp{\text{3}}\)

\item {} 
\sphinxAtStartPar
\sphinxstyleliteralstrong{\sphinxupquote{flight\_speed}} (\sphinxstyleliteralemphasis{\sphinxupquote{float}}) \textendash{} Flight speed; m/s

\end{itemize}

\item[{Returns}] \leavevmode
\sphinxAtStartPar
Parasite power; W

\item[{Return type}] \leavevmode
\sphinxAtStartPar
float

\end{description}\end{quote}

\end{fulllineitems}


\end{fulllineitems}



\section{Helicopter}
\label{\detokenize{modules/helicopter:helicopter}}\label{\detokenize{modules/helicopter::doc}}

\subsection{Sizing loop}
\label{\detokenize{modules/helicopter:sizing-loop}}
\begin{figure}[htbp]
\centering

\noindent\sphinxincludegraphics[width=350\sphinxpxdimen]{{sizing_loop}.png}
\end{figure}
\phantomsection\label{\detokenize{modules/helicopter:module-helicopter}}\index{module@\spxentry{module}!helicopter@\spxentry{helicopter}}\index{helicopter@\spxentry{helicopter}!module@\spxentry{module}}\index{Helicopter (class in helicopter)@\spxentry{Helicopter}\spxextra{class in helicopter}}

\begin{fulllineitems}
\phantomsection\label{\detokenize{modules/helicopter:helicopter.Helicopter}}\pysiglinewithargsret{\sphinxbfcode{\sphinxupquote{class\DUrole{w}{  }}}\sphinxcode{\sphinxupquote{helicopter.}}\sphinxbfcode{\sphinxupquote{Helicopter}}}{\emph{\DUrole{n}{filename}\DUrole{p}{:}\DUrole{w}{  }\DUrole{n}{str}}}{}
\sphinxAtStartPar
Bases: {\hyperref[\detokenize{modules/aircraft:aircraft.Aircraft}]{\sphinxcrossref{\sphinxcode{\sphinxupquote{aircraft.Aircraft}}}}}

\sphinxAtStartPar
Aircraft sub\sphinxhyphen{}class for conventional helicopter configurations containing
methods for preliminary design and analysis.

\begin{sphinxadmonition}{note}{Note:}
\sphinxAtStartPar
Requires \sphinxstyleemphasis{Main rotor}, \sphinxstyleemphasis{Tail rotor}, \sphinxstyleemphasis{Engines}, \sphinxstyleemphasis{Fuselage}, and
\sphinxstyleemphasis{Landing Gear} section in the configuration file.
\end{sphinxadmonition}
\index{main\_rotor (helicopter.Helicopter attribute)@\spxentry{main\_rotor}\spxextra{helicopter.Helicopter attribute}}

\begin{fulllineitems}
\phantomsection\label{\detokenize{modules/helicopter:helicopter.Helicopter.main_rotor}}\pysigline{\sphinxbfcode{\sphinxupquote{main\_rotor}}}
\sphinxAtStartPar
Main rotor.
\begin{quote}\begin{description}
\item[{Type}] \leavevmode
\sphinxAtStartPar
{\hyperref[\detokenize{modules/rotor:rotor.Rotor}]{\sphinxcrossref{Rotor}}}

\end{description}\end{quote}

\end{fulllineitems}

\index{tail\_rotor (helicopter.Helicopter attribute)@\spxentry{tail\_rotor}\spxextra{helicopter.Helicopter attribute}}

\begin{fulllineitems}
\phantomsection\label{\detokenize{modules/helicopter:helicopter.Helicopter.tail_rotor}}\pysigline{\sphinxbfcode{\sphinxupquote{tail\_rotor}}}
\sphinxAtStartPar
Tail rotor.
\begin{quote}\begin{description}
\item[{Type}] \leavevmode
\sphinxAtStartPar
{\hyperref[\detokenize{modules/rotor:rotor.Rotor}]{\sphinxcrossref{Rotor}}}

\end{description}\end{quote}

\end{fulllineitems}

\index{engines (helicopter.Helicopter attribute)@\spxentry{engines}\spxextra{helicopter.Helicopter attribute}}

\begin{fulllineitems}
\phantomsection\label{\detokenize{modules/helicopter:helicopter.Helicopter.engines}}\pysigline{\sphinxbfcode{\sphinxupquote{engines}}}
\sphinxAtStartPar
Turboshaft engines.
\begin{quote}\begin{description}
\item[{Type}] \leavevmode
\sphinxAtStartPar
{\hyperref[\detokenize{modules/engines:engines.Engines}]{\sphinxcrossref{Engines}}}

\end{description}\end{quote}

\end{fulllineitems}

\index{fuselage (helicopter.Helicopter attribute)@\spxentry{fuselage}\spxextra{helicopter.Helicopter attribute}}

\begin{fulllineitems}
\phantomsection\label{\detokenize{modules/helicopter:helicopter.Helicopter.fuselage}}\pysigline{\sphinxbfcode{\sphinxupquote{fuselage}}}
\sphinxAtStartPar
Fuselage.
\begin{quote}\begin{description}
\item[{Type}] \leavevmode
\sphinxAtStartPar
{\hyperref[\detokenize{modules/fuselage:fuselage.Fuselage}]{\sphinxcrossref{Fuselage}}}

\end{description}\end{quote}

\end{fulllineitems}

\index{landing\_gear (helicopter.Helicopter attribute)@\spxentry{landing\_gear}\spxextra{helicopter.Helicopter attribute}}

\begin{fulllineitems}
\phantomsection\label{\detokenize{modules/helicopter:helicopter.Helicopter.landing_gear}}\pysigline{\sphinxbfcode{\sphinxupquote{landing\_gear}}}
\sphinxAtStartPar
Landing Gear.
\begin{quote}\begin{description}
\item[{Type}] \leavevmode
\sphinxAtStartPar
{\hyperref[\detokenize{modules/landing_gear:landing_gear.LandingGear}]{\sphinxcrossref{LandingGear}}}

\end{description}\end{quote}

\end{fulllineitems}

\index{empty\_weight\_estimation() (helicopter.Helicopter method)@\spxentry{empty\_weight\_estimation()}\spxextra{helicopter.Helicopter method}}

\begin{fulllineitems}
\phantomsection\label{\detokenize{modules/helicopter:helicopter.Helicopter.empty_weight_estimation}}\pysiglinewithargsret{\sphinxbfcode{\sphinxupquote{empty\_weight\_estimation}}}{\emph{\DUrole{n}{fuel\_mass}}, \emph{\DUrole{n}{power}}}{}
\sphinxAtStartPar
Estimate the empty weight components of medium helicopters.
{[}pp.408\sphinxhyphen{}416{]}
\begin{quote}\begin{description}
\item[{Parameters}] \leavevmode\begin{itemize}
\item {} 
\sphinxAtStartPar
\sphinxstyleliteralstrong{\sphinxupquote{fuel\_mass}} (\sphinxstyleliteralemphasis{\sphinxupquote{float}}) \textendash{} Fuel mass; kg, used for fuel tank sizing.

\item {} 
\sphinxAtStartPar
\sphinxstyleliteralstrong{\sphinxupquote{power}} (\sphinxstyleliteralemphasis{\sphinxupquote{float}}) \textendash{} Total power; W, used for propulsion system and instrument sizing.

\end{itemize}

\item[{Returns}] \leavevmode
\sphinxAtStartPar
Dictionary containing the empty weight components main rotor, tail
rotor, fuselage and tail, landing gear, transmission, engines, fuel
tanks, flight control systems, hydraulic systems, avionics,
instruments, furnishing, air conditioning and anti\sphinxhyphen{}ice, loading and
handling, and electrical systems. The sum represents the total
empty weight.

\item[{Return type}] \leavevmode
\sphinxAtStartPar
dict

\end{description}\end{quote}

\end{fulllineitems}

\index{evaluate() (helicopter.Helicopter method)@\spxentry{evaluate()}\spxextra{helicopter.Helicopter method}}

\begin{fulllineitems}
\phantomsection\label{\detokenize{modules/helicopter:helicopter.Helicopter.evaluate}}\pysiglinewithargsret{\sphinxbfcode{\sphinxupquote{evaluate}}}{\emph{\DUrole{n}{mission}\DUrole{p}{:}\DUrole{w}{  }\DUrole{n}{{\hyperref[\detokenize{modules/mission:mission.Mission}]{\sphinxcrossref{mission.Mission}}}}}}{}
\sphinxAtStartPar
Evaluate the flight state, powers, and performance throughout the
mission for a fixed configuration. For each segment, the Gross Weight
at the beginning of it is assumed.
\begin{quote}\begin{description}
\item[{Parameters}] \leavevmode
\sphinxAtStartPar
\sphinxstyleliteralstrong{\sphinxupquote{mission}} ({\hyperref[\detokenize{modules/mission:mission.Mission}]{\sphinxcrossref{\sphinxstyleliteralemphasis{\sphinxupquote{Mission}}}}}) \textendash{} Mission profile.

\item[{Returns}] \leavevmode
\sphinxAtStartPar
Dictionary containing lists of the flight state, power composition,
performance, fuel mass, and gross weight throughout the mission.

\item[{Return type}] \leavevmode
\sphinxAtStartPar
dict

\end{description}\end{quote}

\end{fulllineitems}

\index{flight\_state() (helicopter.Helicopter method)@\spxentry{flight\_state()}\spxextra{helicopter.Helicopter method}}

\begin{fulllineitems}
\phantomsection\label{\detokenize{modules/helicopter:helicopter.Helicopter.flight_state}}\pysiglinewithargsret{\sphinxbfcode{\sphinxupquote{flight\_state}}}{\emph{\DUrole{n}{gross\_weight}}, \emph{\DUrole{n}{density}}, \emph{\DUrole{n}{velocity}}, \emph{\DUrole{n}{gamma}}, \emph{\DUrole{n}{gravity}}}{}
\sphinxAtStartPar
Determine flight state variables.
\begin{quote}\begin{description}
\item[{Parameters}] \leavevmode\begin{itemize}
\item {} 
\sphinxAtStartPar
\sphinxstyleliteralstrong{\sphinxupquote{gross\_weight}} (\sphinxstyleliteralemphasis{\sphinxupquote{float}}) \textendash{} Aircraft mass; kg

\item {} 
\sphinxAtStartPar
\sphinxstyleliteralstrong{\sphinxupquote{density}} (\sphinxstyleliteralemphasis{\sphinxupquote{float}}) \textendash{} Density of the surrounding air; kg/m\(\sp{\text{3}}\)

\item {} 
\sphinxAtStartPar
\sphinxstyleliteralstrong{\sphinxupquote{velocity}} (\sphinxstyleliteralemphasis{\sphinxupquote{float}}) \textendash{} Flight speed; m/s

\item {} 
\sphinxAtStartPar
\sphinxstyleliteralstrong{\sphinxupquote{gamma}} (\sphinxstyleliteralemphasis{\sphinxupquote{float}}) \textendash{} Climb angle; rad

\item {} 
\sphinxAtStartPar
\sphinxstyleliteralstrong{\sphinxupquote{gravity}} (\sphinxstyleliteralemphasis{\sphinxupquote{float}}) \textendash{} Gravitational acceleration; m/s\(\sp{\text{2}}\)

\end{itemize}

\item[{Returns}] \leavevmode
\sphinxAtStartPar
Dictionary containing the density, velocity, climb angle, advance
ratio, drag, angle of attack, thrust, induced velocity, download
factor, blade loading, gravity, and gross weight.

\item[{Return type}] \leavevmode
\sphinxAtStartPar
dict

\end{description}\end{quote}

\end{fulllineitems}

\index{initial\_mtow\_estimation() (helicopter.Helicopter method)@\spxentry{initial\_mtow\_estimation()}\spxextra{helicopter.Helicopter method}}

\begin{fulllineitems}
\phantomsection\label{\detokenize{modules/helicopter:helicopter.Helicopter.initial_mtow_estimation}}\pysiglinewithargsret{\sphinxbfcode{\sphinxupquote{initial\_mtow\_estimation}}}{\emph{\DUrole{n}{duration}}, \emph{\DUrole{n}{payload}}, \emph{\DUrole{n}{crew\_mass}}}{}
\sphinxAtStartPar
Estimate the initial maximum take\sphinxhyphen{}off weight based on the installed
power and total duration. Specific fuel consumption and empty weight
ratio are assumed constant. {[}pp.90\sphinxhyphen{}91{]}
\begin{quote}\begin{description}
\item[{Parameters}] \leavevmode\begin{itemize}
\item {} 
\sphinxAtStartPar
\sphinxstyleliteralstrong{\sphinxupquote{duration}} (\sphinxstyleliteralemphasis{\sphinxupquote{float}}) \textendash{} Mission duration; h

\item {} 
\sphinxAtStartPar
\sphinxstyleliteralstrong{\sphinxupquote{payload}} (\sphinxstyleliteralemphasis{\sphinxupquote{float}}) \textendash{} Payload; kg

\item {} 
\sphinxAtStartPar
\sphinxstyleliteralstrong{\sphinxupquote{crew\_mass}} (\sphinxstyleliteralemphasis{\sphinxupquote{float}}) \textendash{} Crew mass; kg

\end{itemize}

\end{description}\end{quote}

\end{fulllineitems}

\index{mass\_estimation() (helicopter.Helicopter method)@\spxentry{mass\_estimation()}\spxextra{helicopter.Helicopter method}}

\begin{fulllineitems}
\phantomsection\label{\detokenize{modules/helicopter:helicopter.Helicopter.mass_estimation}}\pysiglinewithargsret{\sphinxbfcode{\sphinxupquote{mass\_estimation}}}{\emph{\DUrole{n}{power}}, \emph{\DUrole{n}{fuel\_mass}}, \emph{\DUrole{n}{crew\_mass}}, \emph{\DUrole{n}{payload}}, \emph{\DUrole{n}{use\_ew\_models}}}{}
\sphinxAtStartPar
Determine the maximum take\sphinxhyphen{}off weight composition.
\begin{quote}\begin{description}
\item[{Parameters}] \leavevmode\begin{itemize}
\item {} 
\sphinxAtStartPar
\sphinxstyleliteralstrong{\sphinxupquote{power}} (\sphinxstyleliteralemphasis{\sphinxupquote{float}}) \textendash{} Power; W

\item {} 
\sphinxAtStartPar
\sphinxstyleliteralstrong{\sphinxupquote{fuel\_mass}} (\sphinxstyleliteralemphasis{\sphinxupquote{float}}) \textendash{} Fuel mass; kg

\item {} 
\sphinxAtStartPar
\sphinxstyleliteralstrong{\sphinxupquote{crew\_mass}} (\sphinxstyleliteralemphasis{\sphinxupquote{float}}) \textendash{} Crew mass; kg

\item {} 
\sphinxAtStartPar
\sphinxstyleliteralstrong{\sphinxupquote{payload}} (\sphinxstyleliteralemphasis{\sphinxupquote{float}}) \textendash{} Payload; kg

\item {} 
\sphinxAtStartPar
\sphinxstyleliteralstrong{\sphinxupquote{use\_ew\_models}} (\sphinxstyleliteralemphasis{\sphinxupquote{bool}}) \textendash{} Use empirical empty weight models, otherwise the ratio to the
maximum take\sphinxhyphen{}off weight is assumed constant.

\end{itemize}

\item[{Returns}] \leavevmode
\sphinxAtStartPar
Dictionary containing the empty weight, fuel, payload, and crew
mass. The sum represents the maximum take\sphinxhyphen{}off weight.

\item[{Return type}] \leavevmode
\sphinxAtStartPar
dict

\end{description}\end{quote}

\end{fulllineitems}

\index{performance() (helicopter.Helicopter method)@\spxentry{performance()}\spxextra{helicopter.Helicopter method}}

\begin{fulllineitems}
\phantomsection\label{\detokenize{modules/helicopter:helicopter.Helicopter.performance}}\pysiglinewithargsret{\sphinxbfcode{\sphinxupquote{performance}}}{\emph{\DUrole{n}{powers}\DUrole{p}{:}\DUrole{w}{  }\DUrole{n}{dict}}, \emph{\DUrole{n}{temperature}}, \emph{\DUrole{n}{pressure}}}{}
\sphinxAtStartPar
Determine performance parameters aside from the individual power
components.
\begin{quote}\begin{description}
\item[{Parameters}] \leavevmode\begin{itemize}
\item {} 
\sphinxAtStartPar
\sphinxstyleliteralstrong{\sphinxupquote{powers}} (\sphinxstyleliteralemphasis{\sphinxupquote{dict}}) \textendash{} Output of {\hyperref[\detokenize{modules/helicopter:helicopter.Helicopter.powers}]{\sphinxcrossref{\sphinxcode{\sphinxupquote{powers()}}}}}.

\item {} 
\sphinxAtStartPar
\sphinxstyleliteralstrong{\sphinxupquote{temperature}} (\sphinxstyleliteralemphasis{\sphinxupquote{float}}) \textendash{} Temperature of the surrounding air; K

\item {} 
\sphinxAtStartPar
\sphinxstyleliteralstrong{\sphinxupquote{pressure}} (\sphinxstyleliteralemphasis{\sphinxupquote{float}}) \textendash{} Pressure of the surrounding air; Pa

\end{itemize}

\item[{Returns}] \leavevmode
\sphinxAtStartPar
Dictionary containing the total power, the power requirement at
mean sea level, specific fuel consumption, and the fuel mass flow.

\item[{Return type}] \leavevmode
\sphinxAtStartPar
dict

\end{description}\end{quote}

\end{fulllineitems}

\index{plot\_blade\_loading() (helicopter.Helicopter method)@\spxentry{plot\_blade\_loading()}\spxextra{helicopter.Helicopter method}}

\begin{fulllineitems}
\phantomsection\label{\detokenize{modules/helicopter:helicopter.Helicopter.plot_blade_loading}}\pysiglinewithargsret{\sphinxbfcode{\sphinxupquote{plot\_blade\_loading}}}{\emph{\DUrole{n}{blade\_loading}}, \emph{\DUrole{n}{advance\_ratio}}}{}
\sphinxAtStartPar
Plot the blade loading over the advance ratio, in order to evaluate
the stall margin. {[}p.142{]}
\begin{quote}\begin{description}
\item[{Parameters}] \leavevmode\begin{itemize}
\item {} 
\sphinxAtStartPar
\sphinxstyleliteralstrong{\sphinxupquote{blade\_loading}} (\sphinxstyleliteralemphasis{\sphinxupquote{float}}) \textendash{} Blade loading.

\item {} 
\sphinxAtStartPar
\sphinxstyleliteralstrong{\sphinxupquote{advance\_ratio}} (\sphinxstyleliteralemphasis{\sphinxupquote{float}}) \textendash{} Advance ratio.

\end{itemize}

\end{description}\end{quote}

\end{fulllineitems}

\index{plot\_empty\_weight\_pie() (helicopter.Helicopter method)@\spxentry{plot\_empty\_weight\_pie()}\spxextra{helicopter.Helicopter method}}

\begin{fulllineitems}
\phantomsection\label{\detokenize{modules/helicopter:helicopter.Helicopter.plot_empty_weight_pie}}\pysiglinewithargsret{\sphinxbfcode{\sphinxupquote{plot\_empty\_weight\_pie}}}{\emph{\DUrole{n}{empty\_weight\_parts}\DUrole{p}{:}\DUrole{w}{  }\DUrole{n}{dict}}}{}
\sphinxAtStartPar
Plot the empty weight composition as a pie chart.
\begin{quote}\begin{description}
\item[{Parameters}] \leavevmode
\sphinxAtStartPar
\sphinxstyleliteralstrong{\sphinxupquote{empty\_weight\_parts}} (\sphinxstyleliteralemphasis{\sphinxupquote{dict}}) \textendash{} Output of {\hyperref[\detokenize{modules/helicopter:helicopter.Helicopter.empty_weight_estimation}]{\sphinxcrossref{\sphinxcode{\sphinxupquote{empty\_weight\_estimation()}}}}}.

\end{description}\end{quote}

\end{fulllineitems}

\index{plot\_fuel\_curve() (helicopter.Helicopter method)@\spxentry{plot\_fuel\_curve()}\spxextra{helicopter.Helicopter method}}

\begin{fulllineitems}
\phantomsection\label{\detokenize{modules/helicopter:helicopter.Helicopter.plot_fuel_curve}}\pysiglinewithargsret{\sphinxbfcode{\sphinxupquote{plot\_fuel\_curve}}}{\emph{\DUrole{n}{gross\_weight}}, \emph{\DUrole{n}{density}}, \emph{\DUrole{n}{temperature}}, \emph{\DUrole{n}{pressure}}, \emph{\DUrole{n}{max\_velocity}\DUrole{o}{=}\DUrole{default_value}{None}}}{}
\sphinxAtStartPar
Plot the fuel mass flow and total power over a range of horizontal
flight speeds. Annotations show the velocities for best range and
endurance.
\begin{quote}\begin{description}
\item[{Parameters}] \leavevmode\begin{itemize}
\item {} 
\sphinxAtStartPar
\sphinxstyleliteralstrong{\sphinxupquote{gross\_weight}} (\sphinxstyleliteralemphasis{\sphinxupquote{float}}) \textendash{} Aircraft mass; kg

\item {} 
\sphinxAtStartPar
\sphinxstyleliteralstrong{\sphinxupquote{density}} (\sphinxstyleliteralemphasis{\sphinxupquote{float}}) \textendash{} Density of the surrounding air; kg/m\(\sp{\text{3}}\)

\item {} 
\sphinxAtStartPar
\sphinxstyleliteralstrong{\sphinxupquote{temperature}} (\sphinxstyleliteralemphasis{\sphinxupquote{float}}) \textendash{} Temperature of the surrounding air; K

\item {} 
\sphinxAtStartPar
\sphinxstyleliteralstrong{\sphinxupquote{pressure}} (\sphinxstyleliteralemphasis{\sphinxupquote{float}}) \textendash{} Pressure of the surrounding air; Pa

\item {} 
\sphinxAtStartPar
\sphinxstyleliteralstrong{\sphinxupquote{max\_velocity}} (\sphinxstyleliteralemphasis{\sphinxupquote{float}}\sphinxstyleliteralemphasis{\sphinxupquote{, }}\sphinxstyleliteralemphasis{\sphinxupquote{optional}}) \textendash{} Upper limit for the forward flight speed to be considered in the
plot. If not provided, the limit is defined by the installed power.

\end{itemize}

\end{description}\end{quote}

\end{fulllineitems}

\index{plot\_gross\_weight\_over\_time() (helicopter.Helicopter method)@\spxentry{plot\_gross\_weight\_over\_time()}\spxextra{helicopter.Helicopter method}}

\begin{fulllineitems}
\phantomsection\label{\detokenize{modules/helicopter:helicopter.Helicopter.plot_gross_weight_over_time}}\pysiglinewithargsret{\sphinxbfcode{\sphinxupquote{plot\_gross\_weight\_over\_time}}}{\emph{\DUrole{n}{mission}\DUrole{p}{:}\DUrole{w}{  }\DUrole{n}{{\hyperref[\detokenize{modules/mission:mission.Mission}]{\sphinxcrossref{mission.Mission}}}}}, \emph{\DUrole{n}{fuel\_mass\_per\_segment}\DUrole{p}{:}\DUrole{w}{  }\DUrole{n}{list}}}{}
\sphinxAtStartPar
Plot the Gross Weight and fuel flow over the mission duration.
\begin{quote}\begin{description}
\item[{Parameters}] \leavevmode\begin{itemize}
\item {} 
\sphinxAtStartPar
\sphinxstyleliteralstrong{\sphinxupquote{mission}} ({\hyperref[\detokenize{modules/mission:mission.Mission}]{\sphinxcrossref{\sphinxstyleliteralemphasis{\sphinxupquote{Mission}}}}}) \textendash{} Mission profile.

\item {} 
\sphinxAtStartPar
\sphinxstyleliteralstrong{\sphinxupquote{fuel\_mass\_per\_segment}} (\sphinxstyleliteralemphasis{\sphinxupquote{list}}\sphinxstyleliteralemphasis{\sphinxupquote{{[}}}\sphinxstyleliteralemphasis{\sphinxupquote{float}}\sphinxstyleliteralemphasis{\sphinxupquote{{]}}}) \textendash{} Fuel demand for each segment; kg

\end{itemize}

\end{description}\end{quote}

\end{fulllineitems}

\index{plot\_masses\_pie() (helicopter.Helicopter method)@\spxentry{plot\_masses\_pie()}\spxextra{helicopter.Helicopter method}}

\begin{fulllineitems}
\phantomsection\label{\detokenize{modules/helicopter:helicopter.Helicopter.plot_masses_pie}}\pysiglinewithargsret{\sphinxbfcode{\sphinxupquote{plot\_masses\_pie}}}{\emph{\DUrole{n}{masses}\DUrole{p}{:}\DUrole{w}{  }\DUrole{n}{dict}}}{}
\sphinxAtStartPar
Plot the maximum take\sphinxhyphen{}off weight composition as a pie chart.
\begin{quote}\begin{description}
\item[{Parameters}] \leavevmode
\sphinxAtStartPar
\sphinxstyleliteralstrong{\sphinxupquote{masses}} (\sphinxstyleliteralemphasis{\sphinxupquote{dict}}) \textendash{} Output of {\hyperref[\detokenize{modules/helicopter:helicopter.Helicopter.mass_estimation}]{\sphinxcrossref{\sphinxcode{\sphinxupquote{mass\_estimation()}}}}}.

\end{description}\end{quote}

\end{fulllineitems}

\index{plot\_mtow\_convergence() (helicopter.Helicopter method)@\spxentry{plot\_mtow\_convergence()}\spxextra{helicopter.Helicopter method}}

\begin{fulllineitems}
\phantomsection\label{\detokenize{modules/helicopter:helicopter.Helicopter.plot_mtow_convergence}}\pysiglinewithargsret{\sphinxbfcode{\sphinxupquote{plot\_mtow\_convergence}}}{\emph{\DUrole{n}{mtow\_list}\DUrole{p}{:}\DUrole{w}{  }\DUrole{n}{list}}}{}
\sphinxAtStartPar
Plot the maximum take\sphinxhyphen{}off weight over the iterations.
\begin{quote}\begin{description}
\item[{Parameters}] \leavevmode
\sphinxAtStartPar
\sphinxstyleliteralstrong{\sphinxupquote{mtow\_list}} (\sphinxstyleliteralemphasis{\sphinxupquote{list}}\sphinxstyleliteralemphasis{\sphinxupquote{{[}}}\sphinxstyleliteralemphasis{\sphinxupquote{float}}\sphinxstyleliteralemphasis{\sphinxupquote{{]}}}) \textendash{} List of the maximum take\sphinxhyphen{}off weight; kg

\end{description}\end{quote}

\end{fulllineitems}

\index{plot\_power\_curves() (helicopter.Helicopter method)@\spxentry{plot\_power\_curves()}\spxextra{helicopter.Helicopter method}}

\begin{fulllineitems}
\phantomsection\label{\detokenize{modules/helicopter:helicopter.Helicopter.plot_power_curves}}\pysiglinewithargsret{\sphinxbfcode{\sphinxupquote{plot\_power\_curves}}}{\emph{\DUrole{n}{gross\_weight}}, \emph{\DUrole{n}{density}}, \emph{\DUrole{n}{max\_velocity}\DUrole{o}{=}\DUrole{default_value}{None}}}{}
\sphinxAtStartPar
Plot the individual power components over a range of horizontal flight
speeds.
\begin{quote}\begin{description}
\item[{Parameters}] \leavevmode\begin{itemize}
\item {} 
\sphinxAtStartPar
\sphinxstyleliteralstrong{\sphinxupquote{gross\_weight}} (\sphinxstyleliteralemphasis{\sphinxupquote{float}}) \textendash{} Aircraft mass; kg

\item {} 
\sphinxAtStartPar
\sphinxstyleliteralstrong{\sphinxupquote{density}} (\sphinxstyleliteralemphasis{\sphinxupquote{float}}) \textendash{} Density of the surrounding air; kg/m\(\sp{\text{3}}\)

\item {} 
\sphinxAtStartPar
\sphinxstyleliteralstrong{\sphinxupquote{max\_velocity}} (\sphinxstyleliteralemphasis{\sphinxupquote{float}}\sphinxstyleliteralemphasis{\sphinxupquote{, }}\sphinxstyleliteralemphasis{\sphinxupquote{optional}}) \textendash{} Upper limit for the forward flight speed to be considered in the
plot. If not provided, the limit is defined by the installed power.

\end{itemize}

\end{description}\end{quote}

\end{fulllineitems}

\index{plot\_power\_sweep\_drag\_area() (helicopter.Helicopter method)@\spxentry{plot\_power\_sweep\_drag\_area()}\spxextra{helicopter.Helicopter method}}

\begin{fulllineitems}
\phantomsection\label{\detokenize{modules/helicopter:helicopter.Helicopter.plot_power_sweep_drag_area}}\pysiglinewithargsret{\sphinxbfcode{\sphinxupquote{plot\_power\_sweep\_drag\_area}}}{\emph{\DUrole{n}{drag\_area\_range}\DUrole{p}{:}\DUrole{w}{  }\DUrole{n}{tuple}}, \emph{\DUrole{n}{gross\_weight}}, \emph{\DUrole{n}{density}}, \emph{\DUrole{n}{max\_velocity}\DUrole{o}{=}\DUrole{default_value}{75}}}{}
\sphinxAtStartPar
Plot power curves for a range of drag areas.
\begin{quote}\begin{description}
\item[{Parameters}] \leavevmode\begin{itemize}
\item {} 
\sphinxAtStartPar
\sphinxstyleliteralstrong{\sphinxupquote{drag\_area\_range}} (\sphinxstyleliteralemphasis{\sphinxupquote{tuple}}\sphinxstyleliteralemphasis{\sphinxupquote{(}}\sphinxstyleliteralemphasis{\sphinxupquote{min}}\sphinxstyleliteralemphasis{\sphinxupquote{, }}\sphinxstyleliteralemphasis{\sphinxupquote{max}}\sphinxstyleliteralemphasis{\sphinxupquote{, }}\sphinxstyleliteralemphasis{\sphinxupquote{N}}\sphinxstyleliteralemphasis{\sphinxupquote{)}}) \textendash{} Drag area range; kg

\item {} 
\sphinxAtStartPar
\sphinxstyleliteralstrong{\sphinxupquote{gross\_weight}} (\sphinxstyleliteralemphasis{\sphinxupquote{float}}) \textendash{} Aircraft mass; kg

\item {} 
\sphinxAtStartPar
\sphinxstyleliteralstrong{\sphinxupquote{density}} (\sphinxstyleliteralemphasis{\sphinxupquote{float}}) \textendash{} Density of the surrounding air; kg/m\(\sp{\text{3}}\)

\item {} 
\sphinxAtStartPar
\sphinxstyleliteralstrong{\sphinxupquote{max\_velocity}} (\sphinxstyleliteralemphasis{\sphinxupquote{float}}\sphinxstyleliteralemphasis{\sphinxupquote{, }}\sphinxstyleliteralemphasis{\sphinxupquote{optional}}) \textendash{} Maximum velocity to be considered in the plot; m/s

\end{itemize}

\end{description}\end{quote}

\end{fulllineitems}

\index{plot\_power\_sweep\_gw() (helicopter.Helicopter method)@\spxentry{plot\_power\_sweep\_gw()}\spxextra{helicopter.Helicopter method}}

\begin{fulllineitems}
\phantomsection\label{\detokenize{modules/helicopter:helicopter.Helicopter.plot_power_sweep_gw}}\pysiglinewithargsret{\sphinxbfcode{\sphinxupquote{plot\_power\_sweep\_gw}}}{\emph{\DUrole{n}{gw\_range}\DUrole{p}{:}\DUrole{w}{  }\DUrole{n}{tuple}}, \emph{\DUrole{n}{density}}, \emph{\DUrole{n}{max\_velocity}\DUrole{o}{=}\DUrole{default_value}{75}}}{}
\sphinxAtStartPar
Plot power curves for a range of Gross Weights.
\begin{quote}\begin{description}
\item[{Parameters}] \leavevmode\begin{itemize}
\item {} 
\sphinxAtStartPar
\sphinxstyleliteralstrong{\sphinxupquote{gw\_range}} (\sphinxstyleliteralemphasis{\sphinxupquote{tuple}}\sphinxstyleliteralemphasis{\sphinxupquote{(}}\sphinxstyleliteralemphasis{\sphinxupquote{min}}\sphinxstyleliteralemphasis{\sphinxupquote{, }}\sphinxstyleliteralemphasis{\sphinxupquote{max}}\sphinxstyleliteralemphasis{\sphinxupquote{, }}\sphinxstyleliteralemphasis{\sphinxupquote{N}}\sphinxstyleliteralemphasis{\sphinxupquote{)}}) \textendash{} Gross Weight range; kg

\item {} 
\sphinxAtStartPar
\sphinxstyleliteralstrong{\sphinxupquote{density}} (\sphinxstyleliteralemphasis{\sphinxupquote{float}}) \textendash{} Density of the surrounding air; kg/m\(\sp{\text{3}}\)

\item {} 
\sphinxAtStartPar
\sphinxstyleliteralstrong{\sphinxupquote{max\_velocity}} (\sphinxstyleliteralemphasis{\sphinxupquote{float}}\sphinxstyleliteralemphasis{\sphinxupquote{, }}\sphinxstyleliteralemphasis{\sphinxupquote{optional}}) \textendash{} Maximum velocity to be considered in the plot; m/s

\end{itemize}

\end{description}\end{quote}

\end{fulllineitems}

\index{plot\_power\_sweep\_height() (helicopter.Helicopter method)@\spxentry{plot\_power\_sweep\_height()}\spxextra{helicopter.Helicopter method}}

\begin{fulllineitems}
\phantomsection\label{\detokenize{modules/helicopter:helicopter.Helicopter.plot_power_sweep_height}}\pysiglinewithargsret{\sphinxbfcode{\sphinxupquote{plot\_power\_sweep\_height}}}{\emph{\DUrole{n}{height\_range}\DUrole{p}{:}\DUrole{w}{  }\DUrole{n}{tuple}}, \emph{\DUrole{n}{gross\_weight}}, \emph{\DUrole{n}{max\_velocity}\DUrole{o}{=}\DUrole{default_value}{75}}}{}
\sphinxAtStartPar
Plot power curves for a range of heights.
\begin{quote}\begin{description}
\item[{Parameters}] \leavevmode\begin{itemize}
\item {} 
\sphinxAtStartPar
\sphinxstyleliteralstrong{\sphinxupquote{height\_range}} (\sphinxstyleliteralemphasis{\sphinxupquote{tuple}}\sphinxstyleliteralemphasis{\sphinxupquote{(}}\sphinxstyleliteralemphasis{\sphinxupquote{min}}\sphinxstyleliteralemphasis{\sphinxupquote{, }}\sphinxstyleliteralemphasis{\sphinxupquote{max}}\sphinxstyleliteralemphasis{\sphinxupquote{, }}\sphinxstyleliteralemphasis{\sphinxupquote{N}}\sphinxstyleliteralemphasis{\sphinxupquote{)}}) \textendash{} Height range; kg

\item {} 
\sphinxAtStartPar
\sphinxstyleliteralstrong{\sphinxupquote{gross\_weight}} (\sphinxstyleliteralemphasis{\sphinxupquote{float}}) \textendash{} Aircraft mass; kg

\item {} 
\sphinxAtStartPar
\sphinxstyleliteralstrong{\sphinxupquote{max\_velocity}} (\sphinxstyleliteralemphasis{\sphinxupquote{float}}\sphinxstyleliteralemphasis{\sphinxupquote{, }}\sphinxstyleliteralemphasis{\sphinxupquote{optional}}) \textendash{} Maximum velocity to be considered in the plot; m/s

\end{itemize}

\end{description}\end{quote}

\end{fulllineitems}

\index{plot\_powers\_pie() (helicopter.Helicopter method)@\spxentry{plot\_powers\_pie()}\spxextra{helicopter.Helicopter method}}

\begin{fulllineitems}
\phantomsection\label{\detokenize{modules/helicopter:helicopter.Helicopter.plot_powers_pie}}\pysiglinewithargsret{\sphinxbfcode{\sphinxupquote{plot\_powers\_pie}}}{\emph{\DUrole{n}{powers}\DUrole{p}{:}\DUrole{w}{  }\DUrole{n}{dict}}}{}
\sphinxAtStartPar
Plot the power composition as a pie chart.
\begin{quote}\begin{description}
\item[{Parameters}] \leavevmode
\sphinxAtStartPar
\sphinxstyleliteralstrong{\sphinxupquote{powers}} (\sphinxstyleliteralemphasis{\sphinxupquote{dict}}) \textendash{} Output of {\hyperref[\detokenize{modules/helicopter:helicopter.Helicopter.powers}]{\sphinxcrossref{\sphinxcode{\sphinxupquote{powers()}}}}}.

\end{description}\end{quote}

\end{fulllineitems}

\index{plot\_results() (helicopter.Helicopter method)@\spxentry{plot\_results()}\spxextra{helicopter.Helicopter method}}

\begin{fulllineitems}
\phantomsection\label{\detokenize{modules/helicopter:helicopter.Helicopter.plot_results}}\pysiglinewithargsret{\sphinxbfcode{\sphinxupquote{plot\_results}}}{\emph{\DUrole{n}{df}\DUrole{p}{:}\DUrole{w}{  }\DUrole{n}{pandas.core.frame.DataFrame}}, \emph{\DUrole{n}{param}\DUrole{p}{:}\DUrole{w}{  }\DUrole{n}{str}}}{}
\sphinxAtStartPar
Plot one of the preliminary design parameters as a 3D surface over
rotor radius and chord length.
\begin{quote}\begin{description}
\item[{Parameters}] \leavevmode\begin{itemize}
\item {} 
\sphinxAtStartPar
\sphinxstyleliteralstrong{\sphinxupquote{df}} (\sphinxstyleliteralemphasis{\sphinxupquote{pd.DataFrame}}) \textendash{} Output of {\hyperref[\detokenize{modules/helicopter:helicopter.Helicopter.preliminary_design}]{\sphinxcrossref{\sphinxcode{\sphinxupquote{preliminary\_design()}}}}}.

\item {} 
\sphinxAtStartPar
\sphinxstyleliteralstrong{\sphinxupquote{param}} (\sphinxstyleliteralemphasis{\sphinxupquote{str}}) \textendash{} z\sphinxhyphen{}axis parameter, key of {\hyperref[\detokenize{modules/helicopter:helicopter.Helicopter.sizing_loop}]{\sphinxcrossref{\sphinxcode{\sphinxupquote{sizing\_loop()}}}}} output.

\end{itemize}

\end{description}\end{quote}

\end{fulllineitems}

\index{powers() (helicopter.Helicopter method)@\spxentry{powers()}\spxextra{helicopter.Helicopter method}}

\begin{fulllineitems}
\phantomsection\label{\detokenize{modules/helicopter:helicopter.Helicopter.powers}}\pysiglinewithargsret{\sphinxbfcode{\sphinxupquote{powers}}}{\emph{\DUrole{n}{flight\_state}\DUrole{p}{:}\DUrole{w}{  }\DUrole{n}{dict}}}{}
\sphinxAtStartPar
Determine the power composition in the current flight state.
\begin{quote}\begin{description}
\item[{Parameters}] \leavevmode
\sphinxAtStartPar
\sphinxstyleliteralstrong{\sphinxupquote{flight\_state}} (\sphinxstyleliteralemphasis{\sphinxupquote{dict}}) \textendash{} Output of {\hyperref[\detokenize{modules/helicopter:helicopter.Helicopter.flight_state}]{\sphinxcrossref{\sphinxcode{\sphinxupquote{flight\_state()}}}}}.

\item[{Returns}] \leavevmode
\sphinxAtStartPar
Dictionary containing the induced, profile, climb, parasite, tail
rotor, and accessory power as well as transmission losses. The sum
represents the total power.

\item[{Return type}] \leavevmode
\sphinxAtStartPar
dict

\end{description}\end{quote}

\end{fulllineitems}

\index{preliminary\_design() (helicopter.Helicopter method)@\spxentry{preliminary\_design()}\spxextra{helicopter.Helicopter method}}

\begin{fulllineitems}
\phantomsection\label{\detokenize{modules/helicopter:helicopter.Helicopter.preliminary_design}}\pysiglinewithargsret{\sphinxbfcode{\sphinxupquote{preliminary\_design}}}{\emph{\DUrole{n}{mission}\DUrole{p}{:}\DUrole{w}{  }\DUrole{n}{{\hyperref[\detokenize{modules/mission:mission.Mission}]{\sphinxcrossref{mission.Mission}}}}}, \emph{\DUrole{n}{radius\_range}\DUrole{p}{:}\DUrole{w}{  }\DUrole{n}{tuple}}, \emph{\DUrole{n}{chord\_range}\DUrole{p}{:}\DUrole{w}{  }\DUrole{n}{tuple}}, \emph{\DUrole{n}{conv\_tol}\DUrole{o}{=}\DUrole{default_value}{0.0001}}, \emph{\DUrole{n}{use\_ew\_models}\DUrole{o}{=}\DUrole{default_value}{False}}, \emph{\DUrole{n}{status}\DUrole{o}{=}\DUrole{default_value}{True}}}{}
\sphinxAtStartPar
Generate a dataframe for given ranges of the rotor radius and chord
length, containing maximum take\sphinxhyphen{}off weight, fuel consumption, and
other performance parameters.
\begin{quote}\begin{description}
\item[{Parameters}] \leavevmode\begin{itemize}
\item {} 
\sphinxAtStartPar
\sphinxstyleliteralstrong{\sphinxupquote{mission}} ({\hyperref[\detokenize{modules/mission:mission.Mission}]{\sphinxcrossref{\sphinxstyleliteralemphasis{\sphinxupquote{Mission}}}}}) \textendash{} Mission profile.

\item {} 
\sphinxAtStartPar
\sphinxstyleliteralstrong{\sphinxupquote{radius\_range}} (\sphinxstyleliteralemphasis{\sphinxupquote{tuple}}\sphinxstyleliteralemphasis{\sphinxupquote{(}}\sphinxstyleliteralemphasis{\sphinxupquote{min}}\sphinxstyleliteralemphasis{\sphinxupquote{, }}\sphinxstyleliteralemphasis{\sphinxupquote{max}}\sphinxstyleliteralemphasis{\sphinxupquote{, }}\sphinxstyleliteralemphasis{\sphinxupquote{N}}\sphinxstyleliteralemphasis{\sphinxupquote{)}}) \textendash{} Range of the rotor radius variation; m

\item {} 
\sphinxAtStartPar
\sphinxstyleliteralstrong{\sphinxupquote{chord\_range}} (\sphinxstyleliteralemphasis{\sphinxupquote{tuple}}\sphinxstyleliteralemphasis{\sphinxupquote{(}}\sphinxstyleliteralemphasis{\sphinxupquote{min}}\sphinxstyleliteralemphasis{\sphinxupquote{, }}\sphinxstyleliteralemphasis{\sphinxupquote{max}}\sphinxstyleliteralemphasis{\sphinxupquote{, }}\sphinxstyleliteralemphasis{\sphinxupquote{M}}\sphinxstyleliteralemphasis{\sphinxupquote{)}}) \textendash{} Range of the chord length variation; m

\item {} 
\sphinxAtStartPar
\sphinxstyleliteralstrong{\sphinxupquote{conv\_tol}} (\sphinxstyleliteralemphasis{\sphinxupquote{float}}\sphinxstyleliteralemphasis{\sphinxupquote{, }}\sphinxstyleliteralemphasis{\sphinxupquote{optional}}) \textendash{} Convergence tolerance of the sizing loop.

\item {} 
\sphinxAtStartPar
\sphinxstyleliteralstrong{\sphinxupquote{use\_ew\_models}} (\sphinxstyleliteralemphasis{\sphinxupquote{bool}}\sphinxstyleliteralemphasis{\sphinxupquote{, }}\sphinxstyleliteralemphasis{\sphinxupquote{optional}}) \textendash{} Use empirical empty weight models, otherwise the ratio to the
maximum take\sphinxhyphen{}off weight is assumed constant.

\item {} 
\sphinxAtStartPar
\sphinxstyleliteralstrong{\sphinxupquote{status}} (\sphinxstyleliteralemphasis{\sphinxupquote{bool}}\sphinxstyleliteralemphasis{\sphinxupquote{, }}\sphinxstyleliteralemphasis{\sphinxupquote{optional}}) \textendash{} Print status updates, default True.

\end{itemize}

\item[{Returns}] \leavevmode
\sphinxAtStartPar
Dataframe containing the results of {\hyperref[\detokenize{modules/helicopter:helicopter.Helicopter.sizing_loop}]{\sphinxcrossref{\sphinxcode{\sphinxupquote{sizing\_loop()}}}}} as rows.

\item[{Return type}] \leavevmode
\sphinxAtStartPar
pd.DataFrame

\end{description}\end{quote}

\end{fulllineitems}

\index{sizing\_loop() (helicopter.Helicopter method)@\spxentry{sizing\_loop()}\spxextra{helicopter.Helicopter method}}

\begin{fulllineitems}
\phantomsection\label{\detokenize{modules/helicopter:helicopter.Helicopter.sizing_loop}}\pysiglinewithargsret{\sphinxbfcode{\sphinxupquote{sizing\_loop}}}{\emph{\DUrole{n}{mission}\DUrole{p}{:}\DUrole{w}{  }\DUrole{n}{{\hyperref[\detokenize{modules/mission:mission.Mission}]{\sphinxcrossref{mission.Mission}}}}}, \emph{\DUrole{n}{conv\_tol}}, \emph{\DUrole{n}{use\_ew\_models}}}{}
\sphinxAtStartPar
Determine the maximum take\sphinxhyphen{}off weight and performance iteratively for a
given mission.
\begin{quote}\begin{description}
\item[{Parameters}] \leavevmode\begin{itemize}
\item {} 
\sphinxAtStartPar
\sphinxstyleliteralstrong{\sphinxupquote{mission}} ({\hyperref[\detokenize{modules/mission:mission.Mission}]{\sphinxcrossref{\sphinxstyleliteralemphasis{\sphinxupquote{Mission}}}}}) \textendash{} Mission profile.

\item {} 
\sphinxAtStartPar
\sphinxstyleliteralstrong{\sphinxupquote{conv\_tol}} (\sphinxstyleliteralemphasis{\sphinxupquote{float}}) \textendash{} Convergence tolerance.

\item {} 
\sphinxAtStartPar
\sphinxstyleliteralstrong{\sphinxupquote{use\_ew\_models}} (\sphinxstyleliteralemphasis{\sphinxupquote{bool}}) \textendash{} Use empirical empty weight models, otherwise the ratio to the
maximum take\sphinxhyphen{}off weight is assumed constant.

\end{itemize}

\item[{Returns}] \leavevmode
\sphinxAtStartPar
Dictionary containing the radius, chord length, maximum take\sphinxhyphen{}off
weight, fuel mass, empty weight ratio, solidity, disc loading, and
blade loading.

\item[{Return type}] \leavevmode
\sphinxAtStartPar
dict

\end{description}\end{quote}

\end{fulllineitems}


\end{fulllineitems}



\section{Landing Gear}
\label{\detokenize{modules/landing_gear:module-landing_gear}}\label{\detokenize{modules/landing_gear:landing-gear}}\label{\detokenize{modules/landing_gear::doc}}\index{module@\spxentry{module}!landing\_gear@\spxentry{landing\_gear}}\index{landing\_gear@\spxentry{landing\_gear}!module@\spxentry{module}}\index{LandingGear (class in landing\_gear)@\spxentry{LandingGear}\spxextra{class in landing\_gear}}

\begin{fulllineitems}
\phantomsection\label{\detokenize{modules/landing_gear:landing_gear.LandingGear}}\pysiglinewithargsret{\sphinxbfcode{\sphinxupquote{class\DUrole{w}{  }}}\sphinxcode{\sphinxupquote{landing\_gear.}}\sphinxbfcode{\sphinxupquote{LandingGear}}}{\emph{\DUrole{n}{landing\_gear\_data}\DUrole{p}{:}\DUrole{w}{  }\DUrole{n}{dict}}}{}
\sphinxAtStartPar
Bases: \sphinxcode{\sphinxupquote{object}}

\sphinxAtStartPar
Landing gear as an aircraft component.
\index{type\_ (landing\_gear.LandingGear attribute)@\spxentry{type\_}\spxextra{landing\_gear.LandingGear attribute}}

\begin{fulllineitems}
\phantomsection\label{\detokenize{modules/landing_gear:landing_gear.LandingGear.type_}}\pysigline{\sphinxbfcode{\sphinxupquote{type\_}}}
\sphinxAtStartPar
Type of landing gear.
\begin{quote}\begin{description}
\item[{Type}] \leavevmode
\sphinxAtStartPar
str

\end{description}\end{quote}

\end{fulllineitems}

\index{number\_of\_legs (landing\_gear.LandingGear attribute)@\spxentry{number\_of\_legs}\spxextra{landing\_gear.LandingGear attribute}}

\begin{fulllineitems}
\phantomsection\label{\detokenize{modules/landing_gear:landing_gear.LandingGear.number_of_legs}}\pysigline{\sphinxbfcode{\sphinxupquote{number\_of\_legs}}}
\sphinxAtStartPar
Number of legs.
\begin{quote}\begin{description}
\item[{Type}] \leavevmode
\sphinxAtStartPar
int

\end{description}\end{quote}

\end{fulllineitems}


\end{fulllineitems}



\section{Mission}
\label{\detokenize{modules/mission:module-mission}}\label{\detokenize{modules/mission:mission}}\label{\detokenize{modules/mission::doc}}\index{module@\spxentry{module}!mission@\spxentry{mission}}\index{mission@\spxentry{mission}!module@\spxentry{module}}\index{Mission (class in mission)@\spxentry{Mission}\spxextra{class in mission}}

\begin{fulllineitems}
\phantomsection\label{\detokenize{modules/mission:mission.Mission}}\pysiglinewithargsret{\sphinxbfcode{\sphinxupquote{class\DUrole{w}{  }}}\sphinxcode{\sphinxupquote{mission.}}\sphinxbfcode{\sphinxupquote{Mission}}}{\emph{\DUrole{n}{filename}\DUrole{p}{:}\DUrole{w}{  }\DUrole{n}{str}}}{}
\sphinxAtStartPar
Bases: \sphinxcode{\sphinxupquote{object}}

\sphinxAtStartPar
Mission profile.
\index{name (mission.Mission attribute)@\spxentry{name}\spxextra{mission.Mission attribute}}

\begin{fulllineitems}
\phantomsection\label{\detokenize{modules/mission:mission.Mission.name}}\pysigline{\sphinxbfcode{\sphinxupquote{name}}}
\sphinxAtStartPar
Name of the mission.
\begin{quote}\begin{description}
\item[{Type}] \leavevmode
\sphinxAtStartPar
str

\end{description}\end{quote}

\end{fulllineitems}

\index{duration (mission.Mission attribute)@\spxentry{duration}\spxextra{mission.Mission attribute}}

\begin{fulllineitems}
\phantomsection\label{\detokenize{modules/mission:mission.Mission.duration}}\pysigline{\sphinxbfcode{\sphinxupquote{duration}}}
\sphinxAtStartPar
Duration of each segment; h
\begin{quote}\begin{description}
\item[{Type}] \leavevmode
\sphinxAtStartPar
list{[}float{]}

\end{description}\end{quote}

\end{fulllineitems}

\index{payload (mission.Mission attribute)@\spxentry{payload}\spxextra{mission.Mission attribute}}

\begin{fulllineitems}
\phantomsection\label{\detokenize{modules/mission:mission.Mission.payload}}\pysigline{\sphinxbfcode{\sphinxupquote{payload}}}
\sphinxAtStartPar
Payload; kg
\begin{quote}\begin{description}
\item[{Type}] \leavevmode
\sphinxAtStartPar
list{[}float{]}

\end{description}\end{quote}

\end{fulllineitems}

\index{crew\_mass (mission.Mission attribute)@\spxentry{crew\_mass}\spxextra{mission.Mission attribute}}

\begin{fulllineitems}
\phantomsection\label{\detokenize{modules/mission:mission.Mission.crew_mass}}\pysigline{\sphinxbfcode{\sphinxupquote{crew\_mass}}}
\sphinxAtStartPar
Crew mass; kg
\begin{quote}\begin{description}
\item[{Type}] \leavevmode
\sphinxAtStartPar
list{[}float{]}

\end{description}\end{quote}

\end{fulllineitems}

\index{flight\_speed (mission.Mission attribute)@\spxentry{flight\_speed}\spxextra{mission.Mission attribute}}

\begin{fulllineitems}
\phantomsection\label{\detokenize{modules/mission:mission.Mission.flight_speed}}\pysigline{\sphinxbfcode{\sphinxupquote{flight\_speed}}}
\sphinxAtStartPar
Flight speed; m/s
\begin{quote}\begin{description}
\item[{Type}] \leavevmode
\sphinxAtStartPar
list{[}float{]}

\end{description}\end{quote}

\end{fulllineitems}

\index{height (mission.Mission attribute)@\spxentry{height}\spxextra{mission.Mission attribute}}

\begin{fulllineitems}
\phantomsection\label{\detokenize{modules/mission:mission.Mission.height}}\pysigline{\sphinxbfcode{\sphinxupquote{height}}}
\sphinxAtStartPar
Height above ground; m, length n + 1 as defined between segments. In
case of climb/descent, the half\sphinxhyphen{}way height is used to calculate
density, temperature, and pressure.
\begin{quote}\begin{description}
\item[{Type}] \leavevmode
\sphinxAtStartPar
list{[}float{]}

\end{description}\end{quote}

\end{fulllineitems}

\index{gravity (mission.Mission attribute)@\spxentry{gravity}\spxextra{mission.Mission attribute}}

\begin{fulllineitems}
\phantomsection\label{\detokenize{modules/mission:mission.Mission.gravity}}\pysigline{\sphinxbfcode{\sphinxupquote{gravity}}}
\sphinxAtStartPar
Gravitational acceleration; m/s\(\sp{\text{2}}\)
\begin{quote}\begin{description}
\item[{Type}] \leavevmode
\sphinxAtStartPar
list{[}float{]}

\end{description}\end{quote}

\end{fulllineitems}

\index{density (mission.Mission attribute)@\spxentry{density}\spxextra{mission.Mission attribute}}

\begin{fulllineitems}
\phantomsection\label{\detokenize{modules/mission:mission.Mission.density}}\pysigline{\sphinxbfcode{\sphinxupquote{density}}}
\sphinxAtStartPar
Density of the surrounding air; kg/m\(\sp{\text{3}}\)
\begin{quote}\begin{description}
\item[{Type}] \leavevmode
\sphinxAtStartPar
list{[}float{]}

\end{description}\end{quote}

\end{fulllineitems}

\index{temperature (mission.Mission attribute)@\spxentry{temperature}\spxextra{mission.Mission attribute}}

\begin{fulllineitems}
\phantomsection\label{\detokenize{modules/mission:mission.Mission.temperature}}\pysigline{\sphinxbfcode{\sphinxupquote{temperature}}}
\sphinxAtStartPar
Temperature of the surrounding air; K
\begin{quote}\begin{description}
\item[{Type}] \leavevmode
\sphinxAtStartPar
list{[}float{]}

\end{description}\end{quote}

\end{fulllineitems}

\index{pressure (mission.Mission attribute)@\spxentry{pressure}\spxextra{mission.Mission attribute}}

\begin{fulllineitems}
\phantomsection\label{\detokenize{modules/mission:mission.Mission.pressure}}\pysigline{\sphinxbfcode{\sphinxupquote{pressure}}}
\sphinxAtStartPar
Pressure of the surrounding air; Pa
\begin{quote}\begin{description}
\item[{Type}] \leavevmode
\sphinxAtStartPar
list{[}float{]}

\end{description}\end{quote}

\end{fulllineitems}

\index{climb\_angle (mission.Mission attribute)@\spxentry{climb\_angle}\spxextra{mission.Mission attribute}}

\begin{fulllineitems}
\phantomsection\label{\detokenize{modules/mission:mission.Mission.climb_angle}}\pysigline{\sphinxbfcode{\sphinxupquote{climb\_angle}}}
\sphinxAtStartPar
Climb angle; rad
\begin{quote}\begin{description}
\item[{Type}] \leavevmode
\sphinxAtStartPar
list{[}float{]}

\end{description}\end{quote}

\end{fulllineitems}

\index{plot\_mission() (mission.Mission method)@\spxentry{plot\_mission()}\spxextra{mission.Mission method}}

\begin{fulllineitems}
\phantomsection\label{\detokenize{modules/mission:mission.Mission.plot_mission}}\pysiglinewithargsret{\sphinxbfcode{\sphinxupquote{plot\_mission}}}{}{}
\sphinxAtStartPar
Plot the height and payload over the duration of the mission.

\end{fulllineitems}


\end{fulllineitems}

\index{atmosphere() (in module mission)@\spxentry{atmosphere()}\spxextra{in module mission}}

\begin{fulllineitems}
\phantomsection\label{\detokenize{modules/mission:mission.atmosphere}}\pysiglinewithargsret{\sphinxcode{\sphinxupquote{mission.}}\sphinxbfcode{\sphinxupquote{atmosphere}}}{\emph{\DUrole{n}{height}}, \emph{\DUrole{n}{temp\_offset}}}{}
\sphinxAtStartPar
Calculate the density, temperature, and pressure at a given height
based on the international standard atmosphere (ISA). Deviation from
the ISA is considered via a temperature offset. {[}p. 278{]}
\begin{quote}\begin{description}
\item[{Parameters}] \leavevmode\begin{itemize}
\item {} 
\sphinxAtStartPar
\sphinxstyleliteralstrong{\sphinxupquote{height}} (\sphinxstyleliteralemphasis{\sphinxupquote{float}}) \textendash{} Height; m

\item {} 
\sphinxAtStartPar
\sphinxstyleliteralstrong{\sphinxupquote{temp\_offset}} (\sphinxstyleliteralemphasis{\sphinxupquote{float}}) \textendash{} Temperature offset compared to the ISA; K or °C

\end{itemize}

\item[{Returns}] \leavevmode
\sphinxAtStartPar
\begin{itemize}
\item {} 
\sphinxAtStartPar
\sphinxstyleemphasis{float} \textendash{} Density; kg/m\(\sp{\text{3}}\)

\item {} 
\sphinxAtStartPar
\sphinxstyleemphasis{float} \textendash{} Temperature; K

\item {} 
\sphinxAtStartPar
\sphinxstyleemphasis{float} \textendash{} Pressure; Pa

\end{itemize}


\end{description}\end{quote}

\end{fulllineitems}

\index{get\_climb\_angle() (in module mission)@\spxentry{get\_climb\_angle()}\spxextra{in module mission}}

\begin{fulllineitems}
\phantomsection\label{\detokenize{modules/mission:mission.get_climb_angle}}\pysiglinewithargsret{\sphinxcode{\sphinxupquote{mission.}}\sphinxbfcode{\sphinxupquote{get\_climb\_angle}}}{\emph{\DUrole{n}{flight\_speed}}, \emph{\DUrole{n}{climb}}, \emph{\DUrole{n}{duration}}}{}
\sphinxAtStartPar
Calculate the climb angle.
\begin{quote}\begin{description}
\item[{Parameters}] \leavevmode\begin{itemize}
\item {} 
\sphinxAtStartPar
\sphinxstyleliteralstrong{\sphinxupquote{flight\_speed}} (\sphinxstyleliteralemphasis{\sphinxupquote{float}}) \textendash{} Flight speed; m/s

\item {} 
\sphinxAtStartPar
\sphinxstyleliteralstrong{\sphinxupquote{climb}} (\sphinxstyleliteralemphasis{\sphinxupquote{float}}) \textendash{} Vertical distance; m

\item {} 
\sphinxAtStartPar
\sphinxstyleliteralstrong{\sphinxupquote{duration}} (\sphinxstyleliteralemphasis{\sphinxupquote{float}}) \textendash{} Duration; h

\end{itemize}

\item[{Returns}] \leavevmode
\sphinxAtStartPar
Climb angle; rad

\item[{Return type}] \leavevmode
\sphinxAtStartPar
float

\end{description}\end{quote}

\end{fulllineitems}



\section{Rotor}
\label{\detokenize{modules/rotor:module-rotor}}\label{\detokenize{modules/rotor:rotor}}\label{\detokenize{modules/rotor::doc}}\index{module@\spxentry{module}!rotor@\spxentry{rotor}}\index{rotor@\spxentry{rotor}!module@\spxentry{module}}\index{Rotor (class in rotor)@\spxentry{Rotor}\spxextra{class in rotor}}

\begin{fulllineitems}
\phantomsection\label{\detokenize{modules/rotor:rotor.Rotor}}\pysiglinewithargsret{\sphinxbfcode{\sphinxupquote{class\DUrole{w}{  }}}\sphinxcode{\sphinxupquote{rotor.}}\sphinxbfcode{\sphinxupquote{Rotor}}}{\emph{\DUrole{n}{rotor\_data}\DUrole{p}{:}\DUrole{w}{  }\DUrole{n}{dict}}}{}
\sphinxAtStartPar
Bases: \sphinxcode{\sphinxupquote{object}}

\sphinxAtStartPar
Rotors as aircraft components.
\index{radius (rotor.Rotor attribute)@\spxentry{radius}\spxextra{rotor.Rotor attribute}}

\begin{fulllineitems}
\phantomsection\label{\detokenize{modules/rotor:rotor.Rotor.radius}}\pysigline{\sphinxbfcode{\sphinxupquote{radius}}}
\sphinxAtStartPar
Rotor radius; m
\begin{quote}\begin{description}
\item[{Type}] \leavevmode
\sphinxAtStartPar
float

\end{description}\end{quote}

\end{fulllineitems}

\index{number\_of\_blades (rotor.Rotor attribute)@\spxentry{number\_of\_blades}\spxextra{rotor.Rotor attribute}}

\begin{fulllineitems}
\phantomsection\label{\detokenize{modules/rotor:rotor.Rotor.number_of_blades}}\pysigline{\sphinxbfcode{\sphinxupquote{number\_of\_blades}}}
\sphinxAtStartPar
Number of blades.
\begin{quote}\begin{description}
\item[{Type}] \leavevmode
\sphinxAtStartPar
int

\end{description}\end{quote}

\end{fulllineitems}

\index{chord (rotor.Rotor attribute)@\spxentry{chord}\spxextra{rotor.Rotor attribute}}

\begin{fulllineitems}
\phantomsection\label{\detokenize{modules/rotor:rotor.Rotor.chord}}\pysigline{\sphinxbfcode{\sphinxupquote{chord}}}
\sphinxAtStartPar
Chord length; m
\begin{quote}\begin{description}
\item[{Type}] \leavevmode
\sphinxAtStartPar
float

\end{description}\end{quote}

\end{fulllineitems}

\index{kappa (rotor.Rotor attribute)@\spxentry{kappa}\spxextra{rotor.Rotor attribute}}

\begin{fulllineitems}
\phantomsection\label{\detokenize{modules/rotor:rotor.Rotor.kappa}}\pysigline{\sphinxbfcode{\sphinxupquote{kappa}}}
\sphinxAtStartPar
Factor between ideal and real induced power.
\begin{quote}\begin{description}
\item[{Type}] \leavevmode
\sphinxAtStartPar
float

\end{description}\end{quote}

\end{fulllineitems}

\index{zero\_lift\_drag\_coeff (rotor.Rotor attribute)@\spxentry{zero\_lift\_drag\_coeff}\spxextra{rotor.Rotor attribute}}

\begin{fulllineitems}
\phantomsection\label{\detokenize{modules/rotor:rotor.Rotor.zero_lift_drag_coeff}}\pysigline{\sphinxbfcode{\sphinxupquote{zero\_lift\_drag\_coeff}}}
\sphinxAtStartPar
Zero\sphinxhyphen{}lift drag coefficient of the rotor blade, average.
\begin{quote}\begin{description}
\item[{Type}] \leavevmode
\sphinxAtStartPar
float

\end{description}\end{quote}

\end{fulllineitems}

\index{tip\_velocity (rotor.Rotor attribute)@\spxentry{tip\_velocity}\spxextra{rotor.Rotor attribute}}

\begin{fulllineitems}
\phantomsection\label{\detokenize{modules/rotor:rotor.Rotor.tip_velocity}}\pysigline{\sphinxbfcode{\sphinxupquote{tip\_velocity}}}
\sphinxAtStartPar
Tip velocity; m/s
\begin{quote}\begin{description}
\item[{Type}] \leavevmode
\sphinxAtStartPar
float

\end{description}\end{quote}

\end{fulllineitems}

\index{power\_fraction (rotor.Rotor attribute)@\spxentry{power\_fraction}\spxextra{rotor.Rotor attribute}}

\begin{fulllineitems}
\phantomsection\label{\detokenize{modules/rotor:rotor.Rotor.power_fraction}}\pysigline{\sphinxbfcode{\sphinxupquote{power\_fraction}}}
\sphinxAtStartPar
Power fraction relative to the main rotor, if applicable.
\begin{quote}\begin{description}
\item[{Type}] \leavevmode
\sphinxAtStartPar
float

\end{description}\end{quote}

\end{fulllineitems}

\index{installation\_height (rotor.Rotor attribute)@\spxentry{installation\_height}\spxextra{rotor.Rotor attribute}}

\begin{fulllineitems}
\phantomsection\label{\detokenize{modules/rotor:rotor.Rotor.installation_height}}\pysigline{\sphinxbfcode{\sphinxupquote{installation\_height}}}
\sphinxAtStartPar
Distance between landing gear and rotor; m
\begin{quote}\begin{description}
\item[{Type}] \leavevmode
\sphinxAtStartPar
float

\end{description}\end{quote}

\end{fulllineitems}

\index{get\_blade\_loading() (rotor.Rotor method)@\spxentry{get\_blade\_loading()}\spxextra{rotor.Rotor method}}

\begin{fulllineitems}
\phantomsection\label{\detokenize{modules/rotor:rotor.Rotor.get_blade_loading}}\pysiglinewithargsret{\sphinxbfcode{\sphinxupquote{get\_blade\_loading}}}{\emph{\DUrole{n}{density}}, \emph{\DUrole{n}{thrust}}}{}
\sphinxAtStartPar
Calculate the blade loading CT / sigma. {[}p.133{]}
\begin{quote}\begin{description}
\item[{Parameters}] \leavevmode\begin{itemize}
\item {} 
\sphinxAtStartPar
\sphinxstyleliteralstrong{\sphinxupquote{density}} (\sphinxstyleliteralemphasis{\sphinxupquote{float}}) \textendash{} Density of the surrounding air; kg/m\(\sp{\text{3}}\)

\item {} 
\sphinxAtStartPar
\sphinxstyleliteralstrong{\sphinxupquote{thrust}} (\sphinxstyleliteralemphasis{\sphinxupquote{float}}) \textendash{} Thrust; N

\end{itemize}

\item[{Returns}] \leavevmode
\sphinxAtStartPar
Blade loading.

\item[{Return type}] \leavevmode
\sphinxAtStartPar
float

\end{description}\end{quote}

\end{fulllineitems}

\index{get\_climb\_power() (rotor.Rotor method)@\spxentry{get\_climb\_power()}\spxextra{rotor.Rotor method}}

\begin{fulllineitems}
\phantomsection\label{\detokenize{modules/rotor:rotor.Rotor.get_climb_power}}\pysiglinewithargsret{\sphinxbfcode{\sphinxupquote{get\_climb\_power}}}{\emph{\DUrole{n}{flight\_speed}}, \emph{\DUrole{n}{gamma}}, \emph{\DUrole{n}{weight}}}{}
\sphinxAtStartPar
Calculate the climb power due to the change in potential energy.
\begin{quote}\begin{description}
\item[{Parameters}] \leavevmode\begin{itemize}
\item {} 
\sphinxAtStartPar
\sphinxstyleliteralstrong{\sphinxupquote{flight\_speed}} (\sphinxstyleliteralemphasis{\sphinxupquote{float}}) \textendash{} Flight speed; m/s

\item {} 
\sphinxAtStartPar
\sphinxstyleliteralstrong{\sphinxupquote{gamma}} (\sphinxstyleliteralemphasis{\sphinxupquote{float}}) \textendash{} Climb angle; rad

\item {} 
\sphinxAtStartPar
\sphinxstyleliteralstrong{\sphinxupquote{weight}} (\sphinxstyleliteralemphasis{\sphinxupquote{float}}) \textendash{} Aircraft weight; N

\end{itemize}

\item[{Returns}] \leavevmode
\sphinxAtStartPar
Climb power; W

\item[{Return type}] \leavevmode
\sphinxAtStartPar
float

\end{description}\end{quote}

\end{fulllineitems}

\index{get\_disc\_loading() (rotor.Rotor method)@\spxentry{get\_disc\_loading()}\spxextra{rotor.Rotor method}}

\begin{fulllineitems}
\phantomsection\label{\detokenize{modules/rotor:rotor.Rotor.get_disc_loading}}\pysiglinewithargsret{\sphinxbfcode{\sphinxupquote{get\_disc\_loading}}}{\emph{\DUrole{n}{thrust}}}{}
\sphinxAtStartPar
Calculate the disc loading.
\begin{quote}\begin{description}
\item[{Parameters}] \leavevmode
\sphinxAtStartPar
\sphinxstyleliteralstrong{\sphinxupquote{thrust}} (\sphinxstyleliteralemphasis{\sphinxupquote{float}}) \textendash{} Thrust; N

\item[{Returns}] \leavevmode
\sphinxAtStartPar
Disc loading; N/m\(\sp{\text{2}}\)

\item[{Return type}] \leavevmode
\sphinxAtStartPar
float

\end{description}\end{quote}

\end{fulllineitems}

\index{get\_figure\_of\_merit() (rotor.Rotor method)@\spxentry{get\_figure\_of\_merit()}\spxextra{rotor.Rotor method}}

\begin{fulllineitems}
\phantomsection\label{\detokenize{modules/rotor:rotor.Rotor.get_figure_of_merit}}\pysiglinewithargsret{\sphinxbfcode{\sphinxupquote{get\_figure\_of\_merit}}}{\emph{\DUrole{n}{ideal\_induced\_power}}, \emph{\DUrole{n}{profile\_power}}}{}
\sphinxAtStartPar
Calculate the figure of merit.
\begin{quote}\begin{description}
\item[{Parameters}] \leavevmode\begin{itemize}
\item {} 
\sphinxAtStartPar
\sphinxstyleliteralstrong{\sphinxupquote{ideal\_induced\_power}} (\sphinxstyleliteralemphasis{\sphinxupquote{float}}) \textendash{} Ideal induced power; W

\item {} 
\sphinxAtStartPar
\sphinxstyleliteralstrong{\sphinxupquote{profile\_power}} (\sphinxstyleliteralemphasis{\sphinxupquote{float}}) \textendash{} Profile power; W

\end{itemize}

\item[{Returns}] \leavevmode
\sphinxAtStartPar
Figure of merit

\item[{Return type}] \leavevmode
\sphinxAtStartPar
float

\end{description}\end{quote}

\end{fulllineitems}

\index{get\_induced\_velocity() (rotor.Rotor method)@\spxentry{get\_induced\_velocity()}\spxextra{rotor.Rotor method}}

\begin{fulllineitems}
\phantomsection\label{\detokenize{modules/rotor:rotor.Rotor.get_induced_velocity}}\pysiglinewithargsret{\sphinxbfcode{\sphinxupquote{get\_induced\_velocity}}}{\emph{\DUrole{n}{density}}, \emph{\DUrole{n}{v\_inf}}, \emph{\DUrole{n}{alpha}}, \emph{\DUrole{n}{thrust}}}{}
\sphinxAtStartPar
Calculate the induced velocity iteratively in forward flight (not valid
for low sink rate in axial flight). {[}p.253{]}
\begin{quote}\begin{description}
\item[{Parameters}] \leavevmode\begin{itemize}
\item {} 
\sphinxAtStartPar
\sphinxstyleliteralstrong{\sphinxupquote{density}} (\sphinxstyleliteralemphasis{\sphinxupquote{float}}) \textendash{} Density of the surrounding air; kg/m\(\sp{\text{3}}\)

\item {} 
\sphinxAtStartPar
\sphinxstyleliteralstrong{\sphinxupquote{v\_inf}} (\sphinxstyleliteralemphasis{\sphinxupquote{float}}) \textendash{} Inflow velocity; m/s

\item {} 
\sphinxAtStartPar
\sphinxstyleliteralstrong{\sphinxupquote{alpha}} (\sphinxstyleliteralemphasis{\sphinxupquote{float}}) \textendash{} Angle of attack; rad

\item {} 
\sphinxAtStartPar
\sphinxstyleliteralstrong{\sphinxupquote{thrust}} (\sphinxstyleliteralemphasis{\sphinxupquote{float}}) \textendash{} Thrust; N

\end{itemize}

\item[{Returns}] \leavevmode
\sphinxAtStartPar
Induced velocity; m/s

\item[{Return type}] \leavevmode
\sphinxAtStartPar
float

\end{description}\end{quote}

\end{fulllineitems}

\index{get\_min\_power\_radius() (rotor.Rotor method)@\spxentry{get\_min\_power\_radius()}\spxextra{rotor.Rotor method}}

\begin{fulllineitems}
\phantomsection\label{\detokenize{modules/rotor:rotor.Rotor.get_min_power_radius}}\pysiglinewithargsret{\sphinxbfcode{\sphinxupquote{get\_min\_power\_radius}}}{\emph{\DUrole{n}{density}}, \emph{\DUrole{n}{thrust}}}{}
\sphinxAtStartPar
Calculate the optimal rotor radius with respect to induced and profile
power in hover. {[}p.103{]}
\begin{quote}\begin{description}
\item[{Parameters}] \leavevmode\begin{itemize}
\item {} 
\sphinxAtStartPar
\sphinxstyleliteralstrong{\sphinxupquote{density}} (\sphinxstyleliteralemphasis{\sphinxupquote{float}}) \textendash{} Density of the surrounding air; kg/m\(\sp{\text{3}}\)

\item {} 
\sphinxAtStartPar
\sphinxstyleliteralstrong{\sphinxupquote{thrust}} (\sphinxstyleliteralemphasis{\sphinxupquote{float}}) \textendash{} Thrust; N

\end{itemize}

\item[{Returns}] \leavevmode
\sphinxAtStartPar
Optimal radius; m

\item[{Return type}] \leavevmode
\sphinxAtStartPar
float

\end{description}\end{quote}

\end{fulllineitems}

\index{get\_profile\_power() (rotor.Rotor method)@\spxentry{get\_profile\_power()}\spxextra{rotor.Rotor method}}

\begin{fulllineitems}
\phantomsection\label{\detokenize{modules/rotor:rotor.Rotor.get_profile_power}}\pysiglinewithargsret{\sphinxbfcode{\sphinxupquote{get\_profile\_power}}}{\emph{\DUrole{n}{density}}, \emph{\DUrole{n}{advance\_ratio}}}{}
\sphinxAtStartPar
Calculate the profile power. {[}p.258{]}
\begin{quote}\begin{description}
\item[{Parameters}] \leavevmode\begin{itemize}
\item {} 
\sphinxAtStartPar
\sphinxstyleliteralstrong{\sphinxupquote{density}} (\sphinxstyleliteralemphasis{\sphinxupquote{float}}) \textendash{} Density of the surrounding air; kg/m\(\sp{\text{3}}\)

\item {} 
\sphinxAtStartPar
\sphinxstyleliteralstrong{\sphinxupquote{advance\_ratio}} (\sphinxstyleliteralemphasis{\sphinxupquote{float}}) \textendash{} Advance ratio.

\end{itemize}

\item[{Returns}] \leavevmode
\sphinxAtStartPar
Profile power; W

\item[{Return type}] \leavevmode
\sphinxAtStartPar
float

\end{description}\end{quote}

\end{fulllineitems}

\index{in\_ground\_effect() (rotor.Rotor method)@\spxentry{in\_ground\_effect()}\spxextra{rotor.Rotor method}}

\begin{fulllineitems}
\phantomsection\label{\detokenize{modules/rotor:rotor.Rotor.in_ground_effect}}\pysiglinewithargsret{\sphinxbfcode{\sphinxupquote{in\_ground\_effect}}}{\emph{\DUrole{n}{induced\_power}}, \emph{\DUrole{n}{height}}}{}
\sphinxAtStartPar
Calculate the induced power in ground effect (IGE) according to Hayden.
{[}p.288{]}
\begin{quote}\begin{description}
\item[{Parameters}] \leavevmode\begin{itemize}
\item {} 
\sphinxAtStartPar
\sphinxstyleliteralstrong{\sphinxupquote{induced\_power}} (\sphinxstyleliteralemphasis{\sphinxupquote{float}}) \textendash{} Induced power, W

\item {} 
\sphinxAtStartPar
\sphinxstyleliteralstrong{\sphinxupquote{height}} (\sphinxstyleliteralemphasis{\sphinxupquote{float}}) \textendash{} Height of the landing gear above ground; m

\end{itemize}

\item[{Returns}] \leavevmode
\sphinxAtStartPar
Induced power in ground effect; W

\item[{Return type}] \leavevmode
\sphinxAtStartPar
float

\end{description}\end{quote}

\end{fulllineitems}

\index{solidity (rotor.Rotor property)@\spxentry{solidity}\spxextra{rotor.Rotor property}}

\begin{fulllineitems}
\phantomsection\label{\detokenize{modules/rotor:rotor.Rotor.solidity}}\pysigline{\sphinxbfcode{\sphinxupquote{property\DUrole{w}{  }}}\sphinxbfcode{\sphinxupquote{solidity}}}
\sphinxAtStartPar
Rotor solidity (rectangular approximation).

\begin{sphinxadmonition}{note}{Note:}
\sphinxAtStartPar
The solidity is implemented as a property, which means it can be used
as an attribute, calculated on call. This way it will always be in
sync with the rotor radius and chord length.
\end{sphinxadmonition}
\begin{quote}\begin{description}
\item[{Returns}] \leavevmode
\sphinxAtStartPar
Solidity.

\item[{Return type}] \leavevmode
\sphinxAtStartPar
float

\end{description}\end{quote}

\end{fulllineitems}


\end{fulllineitems}



\chapter{Ressources}
\label{\detokenize{ressources:ressources}}\label{\detokenize{ressources::doc}}
\sphinxAtStartPar
This site was created using \sphinxhref{https://www.sphinx-doc.org/en/master/}{Sphinx}, a tool that can generate the documentation for Python modules automatically based on the provided doc\sphinxhyphen{}strings. The following page describes how to add information or create a similar documentation from scratch.


\section{Setup}
\label{\detokenize{ressources:setup}}
\begin{sphinxVerbatim}[commandchars=\\\{\}]
\PYG{n}{pip} \PYG{n}{install} \PYG{n}{sphinx}
\PYG{n}{pip} \PYG{n}{install} \PYG{n}{sphinx\PYGZus{}rtd\PYGZus{}theme}
\end{sphinxVerbatim}

\sphinxAtStartPar
In the \sphinxstyleemphasis{docs} directory:
\begin{enumerate}
\sphinxsetlistlabels{\arabic}{enumi}{enumii}{}{.}%
\item {} 
\sphinxAtStartPar
Initialize Sphinx. (Use default values by repeatedly pressing \sphinxcode{\sphinxupquote{Enter}})

\begin{sphinxVerbatim}[commandchars=\\\{\}]
\PYG{n}{sphinx}\PYG{o}{\PYGZhy{}}\PYG{n}{quickstart}
\end{sphinxVerbatim}

\item {} 
\sphinxAtStartPar
Edit \sphinxstylestrong{conf.py} to set up directories, extentions, themes, etc.

\item {} 
\sphinxAtStartPar
Create the reST files, one for each module and a combined page \sphinxstylestrong{modules.rst}

\begin{sphinxVerbatim}[commandchars=\\\{\}]
\PYG{n}{sphinx}\PYG{o}{\PYGZhy{}}\PYG{n}{apidoc} \PYG{o}{\PYGZhy{}}\PYG{n}{o} \PYG{o}{\PYGZlt{}}\PYG{n}{output\PYGZus{}folder}\PYG{o}{\PYGZgt{}} \PYG{o}{\PYGZlt{}}\PYG{n}{py\PYGZus{}folder}\PYG{o}{\PYGZgt{}}
\end{sphinxVerbatim}

\item {} 
\sphinxAtStartPar
Edit \sphinxstylestrong{index.rst} to include the desired pages

\end{enumerate}


\section{Maintain}
\label{\detokenize{ressources:maintain}}\begin{enumerate}
\sphinxsetlistlabels{\arabic}{enumi}{enumii}{}{.}%
\item {} 
\sphinxAtStartPar
Edit index, modules, or add new pages (in the reStructured\sphinxhyphen{}Text format)

\item {} 
\sphinxAtStartPar
Build HTML and PDF files

\sphinxAtStartPar
In the \sphinxstyleemphasis{docs} directory:

\begin{sphinxVerbatim}[commandchars=\\\{\}]
\PYG{n}{make} \PYG{n}{html}
\PYG{n}{make} \PYG{n}{latexpdf}
\end{sphinxVerbatim}

\end{enumerate}


\section{Guides}
\label{\detokenize{ressources:guides}}\begin{itemize}
\item {} 
\sphinxAtStartPar
\sphinxhref{https://docs.readthedocs.io/en/stable/intro/getting-started-with-sphinx.html}{Getting started with Sphinx}

\item {} 
\sphinxAtStartPar
\sphinxhref{https://draft-edx-style-guide.readthedocs.io/en/latest/ExampleRSTFile.html}{reStructured\sphinxhyphen{}Text}

\end{itemize}


\chapter{Indices and tables}
\label{\detokenize{index:indices-and-tables}}\begin{itemize}
\item {} 
\sphinxAtStartPar
\DUrole{xref,std,std-ref}{genindex}

\item {} 
\sphinxAtStartPar
\DUrole{xref,std,std-ref}{modindex}

\end{itemize}


\renewcommand{\indexname}{Python Module Index}
\begin{sphinxtheindex}
\let\bigletter\sphinxstyleindexlettergroup
\bigletter{a}
\item\relax\sphinxstyleindexentry{aircraft}\sphinxstyleindexpageref{modules/aircraft:\detokenize{module-aircraft}}
\indexspace
\bigletter{e}
\item\relax\sphinxstyleindexentry{engines}\sphinxstyleindexpageref{modules/engines:\detokenize{module-engines}}
\indexspace
\bigletter{f}
\item\relax\sphinxstyleindexentry{fuselage}\sphinxstyleindexpageref{modules/fuselage:\detokenize{module-fuselage}}
\indexspace
\bigletter{h}
\item\relax\sphinxstyleindexentry{helicopter}\sphinxstyleindexpageref{modules/helicopter:\detokenize{module-helicopter}}
\indexspace
\bigletter{l}
\item\relax\sphinxstyleindexentry{landing\_gear}\sphinxstyleindexpageref{modules/landing_gear:\detokenize{module-landing_gear}}
\indexspace
\bigletter{m}
\item\relax\sphinxstyleindexentry{mission}\sphinxstyleindexpageref{modules/mission:\detokenize{module-mission}}
\indexspace
\bigletter{r}
\item\relax\sphinxstyleindexentry{rotor}\sphinxstyleindexpageref{modules/rotor:\detokenize{module-rotor}}
\end{sphinxtheindex}

\renewcommand{\indexname}{Index}
\printindex
\end{document}